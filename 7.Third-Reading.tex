\excnt=1
\chapter{The Third Reading}\label{cha:the_third_reading} 

\chapterprecishere{In this chapter, we will see that quantificational noun
  phrases in the scope of a modal operator can receive a reading where their
  restrictive predicate is not interpreted in the worlds introduced by the modal
  operator (which is what happens in specific readings as well) while \emph{at
    the same time} their quantificational force takes scope below the modal
  operator (which is what happens in non-specific readings as well). This
  seemingly paradoxical situation might force whole-sale revisions to our
  architecture. We discuss the standard solution (which involves supplying
  predicates with world-arguments) and some
  alternatives. % We end with a historical overview: the phenomenon we discuss
                % here has been noted again and again in the history of
                % semantics but there has been surprisingly little
                % cross-fertilization.
}

\minitoc

\section{A Problem}

Janet Dean Fodor discussed examples like (\nextx) in her dissertation
(\citeyear{fodor-diss}).

\ex Mary wanted to buy a hat just like mine. \xe
%
Fodor observes that (\lastx) has three readings, which she labels ``specific
transparent,'' ``non-specific transparent,'' and ``non-specific opaque.''

\begin{enumerate}[(i)] 
\item On the ``specific transparent'' reading, the sentence says that there is a
  particular hat which is just like mine such that Mary has a desire to buy it.
  Say, I am walking along Newbury Street with Mary. Mary sees a hat in a display
  window and wants to buy \emph{it}. She tells me so. I don't reveal that I have
  one just like it. But later I tell \emph{you} by uttering (\lastx).
\item On the ``non-specific opaque'' reading, the sentence says that Mary's
  desire was to buy some hat or other which fulfills the description that it is
  just like mine. She is a copycat.
\item On the ``non-specific transparent'' reading, finally, the sentence will be
  true, e.g., in the following situation: Mary's desire is to buy some hat or
  other, and the only important thing is that it be a Red Sox cap. Unbeknownst
  to her, my hat is one of those as well.
\end{enumerate}
%
The existence of three different readings appears to be problematic for the
scopal account of specific/non-specific ambiguities that we have been assuming.
It seems that our analysis allows just two semantically distinct types of LFs:
Either the DP \emph{a hat just like mine} takes scope below \emph{want}, as in
(\nextx), or it takes scope above \emph{want}, as in (\anextx).

\ex Mary wanted [ [a hat-just-like-mine]$_{1}$ [ \textsc{pro} to buy $t_{1}$ ]] \xe

\ex\ [a hat-just-like-mine]$_{1}$ [ Mary wanted [ \textsc{pro} to buy $t_{1}$ ]] \xe

In the system we have developed so far, (\blastx) says that in every world $w'$
in which Mary gets what she wants, there is something that she buys in $w'$
that's a hat in $w'$ and like my hat in $w'$. This is Fodor's ``non-specific
opaque'' reading. (\lastx), on the other hand, says that there is some thing $x$
which is a hat in the actual world and like my hat in the actual world, and Mary
buys $x$ in every one of her desire worlds. That is Fodor's ``specific
transparent.'' But what about the ``non-specific transparent'' reading? To
obtain this reading, it seems that we would have to evaluate the predicate
\emph{hat just like mine} in the actual world, so as to obtain its actual
extension (in the scenario we have sketched, the set of all Red Sox caps). But
the existential quantifier expressed by the indefinite article in the
\emph{hat}-DP should not take scope over the modal operator \emph{want}, but
below it, so that we can account for the fact that in different desire-worlds of
Mary's, she buys possibly different hats.

There is a tension here: one aspect of the truth-conditions of this reading
suggests that the DP \emph{a hat just like mine} should be \emph{outside} of the
scope of \emph{want}, but another aspect of these truth-conditions compels us to
place it \emph{inside} the scope of \emph{want}. We can't have it both ways, it
would seem, which is why this has been called a ``scope paradox''

Another example of this sort, due to \citet{bauerle:1983:nps}, is (\nextx):

\ex Georg believes that a woman from Stuttgart loves every member of the VfB team. \xe
%
Bäuerle describes the following scenario: Georg has seen a group of men on the
bus. This group happens to be the VfB team (Stuttgart's soccer team), but Georg
does not know this. Georg also believes (Bäuerle doesn't spell out on what
grounds) that there is some woman from Stuttgart who loves every one of these
men. There is no particular woman of whom he believes that, so there are
different such women in his different belief-worlds. Bäuerle notes that (\lastx)
can be understood as true in this scenario. But there is a problem in finding an
appropriate LF that will predict its truth here. First, since there are
different women in different belief-worlds of Georg's, the existential
quantifier \emph{a woman from Stuttgart} must be inside the scope of
\emph{believe}. Second, since (in each belief world) there aren't different
women that love each of the men, but one that loves them all, the \emph{a}-DP
should take scope over the \emph{every}-DP. If the \emph{every}-DP is in the
scope of the \emph{a}-DP, and the \emph{a}-DP is in the scope of \emph{believe},
then it follows that the \emph{every}-DP is in the scope of \emph{believe}. But
on the other hand, if we want to capture the fact that the men in question need
not be VfB-members in Georg's belief-worlds, the predicate \emph{member of the
  VfB team} needs to be outside of the scope of \emph{believe}. Again, we have a
``scope paradox''.

Before we turn to possible solutions for this problem, let's have one
more example:

\ex Mary hopes that a friend of mine will win the race. \xe
%
This again seems to have three readings. In Fodor's terminology, the DP \emph{a
  friend of mine} can be ``non-specific opaque,'' in which case (\lastx) is true
iff in every world where Mary's hopes come true, there is somebody who is my
friend and wins. It can also have a ``specific transparent'' reading: Mary wants
John to win, she doesn't know John is my friend, but I can still report her hope
as in (\lastx). But there is a third option, the ``non-specific transparent''
reading. To bring out this rather exotic reading, imagine this: Mary looks at
the ten contestants and says \emph{I hope one of the three on the right wins -
  they are so shaggy - I like shaggy people}. She doesn't know that those are my
friends. But I could still report her hope as in (\lastx).

\section{The Standard Solution: Overt World Variables}

The scope paradoxes we have encountered can be traced back to a basic design
feature of our system of intensional semantics: the relevant ``evaluation
world'' for each predicate in a sentence is strictly determined by its
LF-position. All predicates that occur in the (immediate) scope of the same
modal operator must be evaluated in the same possible worlds. E.g. if the scope
of \emph{want} consists of the clause \emph{a friend of mine (to) win}, then
every desire-world $w'$ will be required to contain an individual that wins in
$w'$ and is also my friend \emph{in $w'$}. If we want to quantify over
individuals that are my friends in the actual world (and not necessarily in all
the subject's desire worlds), we have no choice but to place \emph{friend of
  mine} outside of the scope of \emph{want}. And if we want to accomplish this
by means of QR, we must move the entire DP \emph{a friend of mine}.

Not every kind of intensional semantics constrains our options in this way. One
way to visualize what we might want is to write down an LF that looks promising:

\ex Mary wanted$_{w_0} [ \lambda w' [$ a hat-just-like-mine $_{w_0}] \lambda x_1 [$ \textsc{pro} to buy$_{w'} x_{1} ]]$ \xe

We have annotated each predicate with the world in which we wish to evaluate it.
$w_0$ is the evaluation world for the entire sentence and it is the world in
which we evaluate the predicates \expression{want} and
\expression{hat-just-like-mine}. The embedded sentence contributes a function
from worlds to truth-values and we insert an explicit $\lambda$-operator binding
the world where the predicate \expression{buy} is evaluated. The crucial aspect
of (\lastx) is that the world in which \expression{hat-just-like-mine} is
evaluated is the matrix evaluation world and not the same world in which its
clause-mate predicate \expression{buy} is evaluated. This LF thus looks like it
might faithfully capture Fodor's third reading.

Logical forms with overt world variables such as (\lastx) are in fact the
standard solution to the problem presented by the third reading. Let us spell
out some of the technicalities. Later, we will consider a couple of
alternatives.

We return to the basic system used in Heim \amp\ Kratzer up to chapter 11. The
interpretation function is relativized only to an assignment function, not to
any other evaluation parameters such as a world, a time, or an index. The
semantic rules are Functional Application, Predicate Abstraction, and Predicate
Modification, in their formulations from the earlier part of H\amp K. There is
no rule of Intensional Functional Application. The only ingredient of
intensional semantics that we do retain is the expanded type system and
ontology. We have a third basic type besides $e$ and $t$, the type $s$. $D_s$ is
the set of all indices, for now possible worlds (later: world-time pairs).

There are a number of innovations in the lexicon and in the syntax. As for the
lexicon, the main change concerns the treatment of predicates (verbs, nouns,
adjectives). They now all get an additional argument, of type $s$.\marginnote{The
  decision to make the world-argument the predicate's first (lowest) argument is
  arbitrary, and nothing hinges on it. For all we know, it could be the highest
  argument, or somewhere in between.}

\pex\label{124}
\a $\sv{\mbox{smart}} = \lambda w\in D_{s}.\ \lambda x\in D_{e}.\ x$ is smart in $w$ 
\a $\sv{\mbox{likes}} = \lambda w\in D_{s}.\ \lambda x\in D_{e}.\ \lambda y\in D_{e}$. $y$ likes $x$ in $w$ 
\a $\sv{\mbox{teacher}} = \lambda w\in D_{s}.\ \lambda x\in D_{e}$. $x$ is a teacher in $w$ 
\a $\sv{\mbox{friend}} = \lambda w\in D_{s}.\ \lambda x\in D_{e}.\ \lambda y\in D_{e}$. $y$ is $x$'s friend in $w$
\xe

This also applies to attitude predicates, modals, and tenses. We illustrate with
\emph{believe} and \emph{must}:

\pex
\a $\sv{\mbox{believe}} = \lambda w\in D_{s}.\ \lambda p\in D_{\angles{s,t}}.\ \lambda x\in D.$ \\
\null\hfill$\forall w'$ [$w'$ conforms to what $x$ believes in $w \rightarrow p(w') = 1$] 
\a $\sv{\mbox{must}} = \lambda w\in D_{s}.\ \lambda R\in D_{\angles{s,st}}.\ \lambda p\in D_{\angles{s,t}}$.\\
\null\hfill$\forall w'\ [R(w)(w') = 1 \rightarrow p(w') = 1]$
\xe
%
Note that predicates (ordinary ones and modal ones), like the ones in \refx{124}
and (\lastx) now have as their semantic values what used to be their
\emph{intensions}.

There is no change to the entries of proper names, determiners, or
truth-functional connectives; these keep their purely extensional (``s-free'')
types and meanings:

\pex
\a $\sv{\mbox{Ann}} = $Ann 
\a $\sv{\mbox{and}} = \lambda u\in D_{t}.\ [ \lambda v\in D_{t}.\ u=v=1]$ 
\a $\sv{\mbox{the}} = \lambda f\in D_{\angles{e,t}}\co \exists !x.\ f(x) = 1.$ the $y$ such that $f(y) = 1$. 
\a $\sv{\mbox{every}} = \lambda f\in D_{\angles{e,t}}.\ \lambda g\in D_{\angles{e,t}}.\ \forall x [ f(x) = 1 \rightarrow g(x) = 1]$
\xe

So, let's start analyzing a simple sentence.

\ex\ [$_{VP}$ John leaves] \xe
%
The verb's type is $\angles{s,et}$, so it's looking for a sister node which
denotes a \emph{world}. \emph{John}, which denotes an individual, is not a
suitable argument. 

We get out of this problem by adding a couple of items to our lexicon, which are
abstract (unpronounced) morphemes. One is a series of pronouns of type $s$
(``index pronouns'' or, for now, ``world pronouns''). In this chapter, we will
write them as $w_n$, with a numerical subscript $n$, or even as $w, w', w''$.
(Later, we sometimes might write them as \emph{pro}$_n$ and rely on context to
make clear we are not referring to an individual.) Their semantics is what you
expect: they get values from the assignment function.

We will stipulate that a complete (matrix) sentence must not contain any free
variables of type $s$ and must receive a denotation of type
\type{s,t}.\footnote{In the 2016 edition of this class, Suzana Fong noted that
  this stipulation is prima facie less appealing than the alternative assumption
  that type-$s$ pronouns are exactly like type-$e$ pronoun in every respect,
  including the ability to remain free and get values from a contextually
  supplied assignment. Irene tried to sketch some principled reason why it might
  not be possible to refer to a specific world other than the world one is in.
  But as Mitya Privoznov pointed out, a similar idea is not plausible for times,
  given the existence of temporal deictics like \emph{then}. So at best there
  might be a principled reason why the world-coordinate of a free index-pronoun
  would always have to be $w_u$. Irene had to concede therefore that the ban
  against free index-pronouns was just a stipulation. We want to think more
  about (a) whether we really need it, and (b) if we do, what might explain it.}
This means that we need binders of world pronouns. Many proposals in this line
of thought help themselves to freely inserted covert binders. We will follow
H\amp K in not doing that. Instead we posit one more lexical item, analogous to
the covert vacuous operator \emph{PRO} of type $e$ in H\amp K (pp.227-228): a
semantically vacuous operator, \emph{OP}, which moves and leaves a trace of type
$s$. Its syntactic properties are such that it must end up in C or right below a
functional head in the ``clausal spine'' between C and V, and it must get there
by a very short movement, a kind of ``head movement''. We are leaving this
rather vague.

So, our sentence \emph{John leaves} contains \emph{OP}, generated as the first
sister of the verb and then moved to the ``top'' of the sentence:

\ex \emph{OP} 1 [ John [ leaves $t_1$ ]] \xe
%
Our system generates the following denotation for (\lastx): ``$\lambda w_s.$
John leaves in $w$'', a proposition. We rewrite the definition of truth/falsity
of an utterance as follows:

\ex An utterance of a sentence (=LF) $\phi$ in world $w$ is true iff
$\sv{\phi}($w$) = 1$. \xe
%
So, if we utter our sentence in this world (call it $w_@$), then the utterance
was true iff John leaves in $w_@$.

Now, we have to look at more complex sentences. First, a simple case of
embedding. The sentence is \expression{John wants to leave}, which now has an LF
like this:

\ex $[\ \emph{OP}\ 1\ [ \mbox{ John } [ \mbox{ wants } t_{1}\ [\ \emph{OP}\ 2\ [
\mbox{ \textsc{pro}(= John) } [ \mbox{ leave } t_{2}\ ]]]]]]$ \xe
 
\begin{exercise}
	Calculate the semantic value of (\lastx). \eex
\end{exercise}

Next, look at an example involving a complex subject, such as \expression{the
  teacher left}. The verb will need a world argument as before. The noun
\expression{teacher} will likewise need one, so that \expression{the} can get
the required argument of type $\angles{e,t}$ (not $\angles{s,et}$!). Now, our
system makes an interesting prediction: one of the world arguments has to be
\emph{OP} and one of them has to be a pronoun. (Why?) We have free choice, it
appears, as to which predicate gets which kind of world argument. Let's assume,
for now, that we insert \emph{OP} as the sister of the verb and a world pronoun
$w_?$ as the sister of the noun. Since we have stipulated that a complete
sentence cannot contain any free world pronouns, the operator and the pronoun
have to be co-indexed. So, after \emph{OP}-movement, we will have this LF:

\ex \emph{OP} 1 [ [ The [ teacher $w_1$ ]] [ left $t_1$ ]] \xe
%
This will denote the correct proposition (true of a world $w$ iff the unique
individual who is a teacher in $w$ left in $w$).

Now comes the payoff. Consider what happens when the sentence contains both a
modal operator and a complex DP in its complement.

\ex \label{fom} Mary wants a friend of mine to win. \xe
%
There are now three predicates that need world arguments. Furthermore, since
\emph{want} needs a proposition as its second argument (after its world
argument), there needs to be an \emph{OP} on top of the embedded clause. There
also needs to be an \emph{OP} on top of the matrix clause. As before, \emph{a
  friend of mine} can stay in the embedded clause or QR into the matrix clause.
When it moves into the matrix clause, the only way to not leave its world
argument free is to co-index it with the matrix \emph{OP}. But when it stays
below, we can choose to co-index it with either \emph{OP}, which is how we
generate two non-specific readings, one opaque and one transparent. Here are the
three LFs (to make the structures more readable, we leave off most of the
bracketing and start writing the world arguments as subscripts to the
predicates):\marginnote{So, instead of writing $t_1$ for a trace of type $s$
  that serves as the first argument of \emph{leaves}, say, we write ``leaves$_{w_1}$''}

\pex
\a non-specific opaque:\\
\emph{OP} 1 Mary wants$_{w_1}$ [ \emph{OP} 2 a friend-of-mine$_{w_2}$ leave$_{w_2}$ ]
\a specific transparent:\\
\emph{OP} 1 a friend-of-mine$_{w_1}$ 3 Mary wants$_{w_1}$ [ \emph{OP} 2 $t_3$ leave$_{w_2}$
]
\a non-specific transparent:\\
\emph{OP} 1 Mary wants$_{w_1}$ [ \emph{OP} 2 a friend-of-mine$_{w_1}$ leave$_{w_2}$ ]
\xe
%
Notice that the third reading is minimally different from the first reading: all
that happened is the choice to co-index the world argument of \emph{friend of
  mine} with the matrix \emph{OP}.

In this new framework, then, we have a way of resolving the apparent ``scope
paradoxes'' and of acknowledging Fodor's point that there are two separate
distinctions to be made when DPs interact with modal operators. First, there is
the scopal relation between the DP and the operator; the DP may take wider scope
(Fodor's ``specific'' reading) or narrower scope (``non-specific'' reading) than
the operator. Second, there is the choice of binder for the world-argument of
the DP's restricting predicate; this may be cobound with the world-argument of
the embedded predicate (Fodor ``opaque'') or with the modal operator's own
world-argument (``transparent''). So the transparent/opaque distinction in the
sense of Fodor is not \emph{per se} a distinction of scope; but it has a
principled connection with scope in one direction: Unless the DP is within the
modal operator's scope, the opaque option (= co-binding the world-pronoun with
the embedded predicate's world-argument) is in principle unavailable. (Hence
``specific'' implies ``transparent'', and ``opaque'' implies ``non-specific''.)
But there is no implication in the other direction: if the DP has narrow scope
w.r.t. to the modal operator, either the local or the long-distance binding
option for its world-pronoun is in principle available. Hence ``non-specific''
readings may be either ``transparent'' or ``opaque''.

% \begin{exercise}\label{yourabstract}
	
% 	For DPs with extensions of type $e$ (specifically, DPs headed by the definite article), there is a truth-conditionally manifest transparent/opaque distinction, but no truth-conditionally detectable specific/non-specific distinction. In other words, if we construct LFs analogous to (\lastx)[a-c] above for an example with a definite DP, we can always prove that the first option (wide scope DP) and the third option (narrow scope DP with distantly bound world-pronoun) denote identical propositions. In this exercise, you are asked to show this for the example in (\nextx).
	
% 	\ex John believes that your abstract will be accepted. \eex \xe

% \end{exercise}

% \subsection{The Need for a Binding Theory for World Variables}

% One could in principle imagine some indexings of our LFs that we have not considered so far. The following LF indexes the predicate of the complement clause to the matrix $\lambda$-operator rather than to the one on top of its own clause.

% \ex $\lambda w_{0} \mbox{ John wants}_{w_{0}} [ \lambda w_{1} \mbox{ \textsc{pro} } \mbox{leave}_{w_{0}} ]$ \xe

% Of course, the resulting semantics would be pathological: what John would be claimed to stand in the wanting relation to is a set of worlds that is either the entire set $W$ of possible worlds (if the evaluation world is one in which John leaves) or the empty set (if the evaluation world is one in which John doesn't leave). Clearly, the sentence has no such meaning. Do we need to restrict our system to not generate such an LF? Perhaps not, if the meaning is so absurd that the LF would be filtered out by some overarching rules distinguishing sense from nonsense.

% But the problem becomes real when we look at more complex examples. Here is one discussed by Percus in important work \citep{percus:other00}:

% \ex Mary thinks that my brother is Canadian. \xe

% Since the subject of the lower clause is a type e expression, we expect at least two readings: opaque and transparent, cf. Exercise \ref{yourabstract}. The two LFs are as follows:

% \pex
% \a opaque\\
% $\lambda w_{0} \mbox{ Mary thinks}_{w_{0}} [ \mbox{ (that) } \lambda w_{1} \mbox{ my brother}_{w_{1}} \mbox{ (is) Canadian}_{w_{1}} ]$
% \a transparent\\
% $\lambda w_{0} \mbox{ Mary thinks}_{w_{0}} [ \mbox{ (that) } \lambda w_{1} \mbox{ my brother}_{w_{0}} \mbox{ (is) Canadian}_{w_{1}} ]$
% \xe

% But as Percus points out, there is another indexing that might be generated:

% \ex $\lambda w_{0} \mbox{ Mary thinks}_{w_{0}} [ \mbox{ (that) } w_{1} \mbox{ my brother}_{w_{1}} \mbox{ (is) Canadian}_{w_{0}} ]$ \xe

% In (\lastx), we have co-indexed the main predicate of the lower clause with the matrix $\lambda$-operator and co-indexed the nominal predicate \expression{brother} with the embedded $\lambda$-operator. That is, in comparison with the transparent reading in (\blastx b), we have just switched around the indices on the two predicates in the lower clause.

% Note that this LF will not lead to a pathological reading. So, is the predicted reading one that the sentence actually has? No. For the transparent reading, we can easily convince ourselves that the sentence does have that reading. Here is Percus' scenario: ``My brother's name is Allon. Suppose Mary thinks Allon is not my brother but she also thinks that Allon is Canadian.'' In such a scenario, our sentence can be judged as true, as predicted if it can have the LF in (\blastx b). But when we try to find evidence that (\lastx) is a possible LF for our sentence, we fail. Here is Percus:

% \begin{quote}
%   If the sentence permitted a structure with this indexing, we would take the sentence to be true whenever there is some \emph{actual} Canadian who \emph{Mary thinks} is my brother \dash even when this person is not my brother in actuality, and \emph{even when Mary mistakenly thinks that he is not Canadian}. For instance, we would take the sentence to be true when Mary thinks that Pierre (the Canadian) is my brother and naturally concludes \dash since she knows that \emph{I} am American \dash that Pierre too is American. But in fact we judge the sentence to be \emph{false} on this scenario, and so there must be something that makes the indexing in (\lastx) impossible. 
% \end{quote}
% %
% Percus then proposes the following descriptive generalization:

% \ex \textsc{Generalization X}: The situation pronoun that a verb selects for must be coindexed with the nearest $\lambda$ above it.\footnote{Percus works with situation pronouns rather than world pronouns, an immaterial difference for our purposes here.} \xe

% We expect that there will need to be a lot of work done to understand the deeper sources of this generalization. For fun, we offer the following implementation (devised by Irene Heim).

% \subsection{Two Kinds of World Pronouns}

% We distinguish two syntactic types of world-pronouns. One type, \emph{$w$-\textsc{pro}}, behaves like relative pronouns and \emph{\textsc{pro}} in the analysis of H\amp K, ch. 8.5 (pp. 226ff.): it is semantically vacuous itself, but can move and leave a trace that is a variable. The only difference between \emph{$w$-\textsc{pro}} and \emph{\textsc{pro}} is that the latter leaves a variable of type $e$ when it moves, whereas the former leaves a variable of type $s$. The other type of world-pronoun, \emph{$w$-pro}, is analogous to bound-variable personal pronouns, i.e., it is itself a variable (here of type $s$). Like a personal pronoun, it can be coindexed with the trace of an existing movement chain.

% With this inventory of world-pronouns, we can capture the essence of Generalization X by stipulating that \emph{$w$-pro} is only generated in the immediate scope of a determiner (i.e., as sister to the determiner's argument). Everywhere else where a world-pronoun is needed for interpretability, we must generate a \emph{$w$-\textsc{pro}} and move it. This (with some tacit assumptions left to the reader to puzzle over) derives the result that the predicates inside nominals can be freely indexed but that the ones inside predicates are captured by the closest $\lambda$-operator.

% As we said, there is plenty more to be explored in the Binding Theory for world pronouns. The reader is referred to the paper by Percus and the references he cites.

% \subsection{Excursus: Semantic reconstruction for non-specific raised subjects?}

% Let us look back at the account of non-specific readings of raised subjects that we sketched earlier
% in Section \ref{sem}. We showed that you can derive such readings by positing a high type trace for the subject raising, a trace of type $\angles{s,\angles{et,t}}$. Before the lower predicate can combine with the trace, the semantic value of the trace has to be extensionalized by being applied to the lower evaluation world (done via the EFA composition principle). Upstairs the raised subject has to be combined with the $\lambda$-abstract (which will be of type $\angles{\angles{s,\angles{et,t}},t}$) via its intension.

% We then saw recently discovered data suggesting that syntactic reconstruction is actually what is going on. This, of course, raises the question of why semantic reconstruction is unavailable (otherwise we wouldn't expect the data that we observed).

% \citet[p. 171, fn. 41]{fox:2000} mentions two possible explanations: 

% \begin{enumerate}[(i)] 
%   \item ``traces, like pronouns, are always interpreted as variables that range over individuals (type $e$)'', 
%   \item ``the semantic type of a trace is determined to be the lowest type compatible with the syntactic environment (as suggested in \citet{beck:diss})''. 
% \end{enumerate}
% %
% In this excursus, we will briefly consider whether our new framework has something to say about this issue. Let's figure out what we would have to do in the new framework to replicate the account in the section on semantics reconstruction.

% Downstairs, we would have a trace of type $\angles{s,\angles{et,t}}$. To calculate its extension, we do not need recourse to a special composition principle, but can simply give it a world-argument (co-indexed with the abstractor resulting from the movement of the $w$-\textsc{pro} in the argument position of the lower verb).

% Now, what has to happen upstairs? Well, there we need the subject to be of type $\angles{s,\angles{et,t}}$, the same type as the trace, to make sure that its semantics will enter the truth-conditions downstairs. But how can we do this?

% We need the DP \emph{somebody from New York} to have as its semantic value an intension, the function from any world to the existential quantifier over individuals who are people from New York in that world. This is actually hard to do in our system. It \emph{would} be possible if (i) the predicate(s) inside the DP received $w$-\textsc{pro} as their argument, and if (ii) that $w$-\textsc{pro} were allowed to moved to adjoin to the DP. If we manage to rule out at least one of the two preconditions on principled grounds, we would have derived the impossibility of semantic reconstruction as a way of getting non-specific readings of raised subjects.

% \begin{enumerate}[(i)] 
%   \item may be ruled out by the Binding Theory for world pronominals, when it gets developed. 
%   \item may be ruled out by principled considerations as well. Perhaps, world-abstractors are only allowed at sentential boundaries. %See \citet{larson:grammar-intensionality} for some discussion of recalcitrant cases, one of which is the object position of so-called intensional transitive verbs, the topic of another section. 
% \end{enumerate}

% \section{Alternatives to Overt World Variables }

% We presented (a variant of) what is currently the most widely accepted solution to the scope paradoxes, which required the use of non-locally bound world-variables. There are some alternatives, one of which is to some extent a ``notational variant'', the others involved syntactic scoping after all.

% \subsection{Indexed Operators}

% It is possible to devise systems where predicates maintain the semantics we originally gave them, according to which they are sensitive to a world of evaluation parameter. The freedom needed to account for the third reading and further facts would be created by assuming more sophisticated operators that shift the evaluation world. Here is a toy example:

% \ex Mary wants [ a [ \textsc{actually}$_0$ friend-of-mine ] leave ] \xe

% The idea is that the \textsc{actually} ``temporarily'' shifts the evaluation world back to what it was ``before'' the abstraction over worlds triggered by \expression{want} happened.

% This kind of system can be spelled out in as much detail as the world-variable analysis. \citet{cresswell:entities} proves that the two systems are equivalent in their expressive power. The decision is therefore a syntactic one. Does natural language have a multitude of indexed world-shifters or a multitude of indexed world-variables? Cresswell suspects the former, as did \citet{kamp:now} who wrote:

% \begin{quote}
%   I of course exclude the possibility of symbolizing the sentence by means of explicit quantification over moments. Such a symbolization would certainly be possible; and it would even make the operators P and F superfluous. Such symbolizations, however, are a considerable departure from the actual form of the original sentences which they represent \dash which is unsatisfactory if we want to gain insight into the semantics of English. Moreover, one can object to symbolizations involving quantification over such abstract objects as moments, if these objects are not explicitly mentioned in the sentences that are to be symbolized. 
% \end{quote}
% %
% There is some resistance to world-time variables because they are not phonetically realized. But in an operator-based system, we'll have non-overt operators all over the place. So, there is no a priori advantage for either system. We will stick with the more transparent LFs with world variables.

% \subsection{Scoping After All?}

% Suppose we didn't give up our previous framework, in which the evaluation-world for any predicate was strictly determined by its LF-position. It turns out that there is a way (actually, two ways) to derive Fodor's non-specific transparent reading in that framework after all.

% Recall again what we need. We need a way to evaluate the restrictive predicate of a DP with respect to the higher evaluation world while at the same time interpreting the quantificational force of the DP downstairs in its local clause. We saw that if we move the DP upstairs, we get the restriction evaluated upstairs but we also have removed the quantifier from where it should exert its force. And if we leave the DP downstairs where its quantificational forces is felt, its restriction is automatically evaluated down there as well. That is why Fodor's reading is paradoxical for the old framework. In fact, though there is no paradox.

% \begin{enumerate}[{Way} 1]
  
%   \item Raise the DP upstairs but leave a $\angles{\angles{e,t},t}$ trace. This way the restriction is evaluated upstairs, then a quantifier extension is calculated, and that quantifier extension is transmitted to trace position. This is just what we needed.
  
%   \item Move the NP-complement of a quantificational D independently of the containing DP.\footnote{Something like this was proposed by \citet{groenendijk-stokhof:L&P:82} in their treatment of questions with \emph{which}-DPs.} Then we could generate three distinct LFs for a sentence like \expression{Mary wants a friend of mine to win}: two familiar ones, in which the whole DP \emph{a friend of mine} is respectively inside and outside the scope of \emph{want}, plus a third one, in which the NP \emph{friend of mine} is outside the scope of \emph{want} but the remnant DP \emph{a} [$_{NP}$ \emph{t}] has been left behind inside it:
  
%   \ex \label{138} [ [NP f-o-m] $\lambda_1$ [ Mary [ want [ [DP a
%   $t_{\angles{e,t},1}$ ] win]]]] \xe
  
%   \begin{exercise}
%       Convince yourself that this third LF represents the narrow-quantification, restrictor-transparent reading (Fodor's ``non-specific transparent"). \eex 
%   \end{exercise}
% \end{enumerate}
% %
% We have found, then, that it is in principle possible after all to account for narrow-Q transparent readings within our original framework of intensional semantics.

% \begin{exercise}
%   In \refx{138}, we chose to annotate the trace of the movement of the NP with the type-label $\angles{e,t}$, thus treating it as a variable whose values are predicate-extensions (characteristic functions of sets of individuals). As we just saw, this choice led to an interpretable structure. But was it our only possible choice? Suppose the LF-structure were exactly as in \refx{138}, except that the trace had been assigned type $\angles{s,et}$ instead of $\angles{e,t}$. Would the tree still be interpretable? If yes, what reading of the sentence would it express? \eex 
% \end{exercise}

% \begin{exercise}
%   We noted in the previous section about the world-pronouns framework that there was a principled reason why restrictor-\emph{de dicto} readings necessarily are narrow-quantification readings. (Or, in Fodor's terms, why there is no such thing as a ``specific \emph{de dicto}'' reading.) In that framework, this was simply a consequence of the fact that bound variables must be in the scope of their binders. What about the alternative account that we have sketched in the present section? Does this account also imply that opaque readings are necessarily narrow-Q? \eex 
% \end{exercise}

% \section{Scope, Restrictors, and the Syntax of Movement}

% To conclude our discussion of the ambiguities of DPs in the complements of modal operators, let us consider some implications for the study of LF-syntax. This will be very inconclusive.

% Accepting the empirical evidence for the existence of narrow-Q transparent readings which are truth-conditionally distinct from both the wide-Q transparent and the narrow-Q opaque readings, we are facing a choice between two types of theories. One theory, which we have referred to as the ``standard'' one, uses a combination of DP-movement and world-pronoun binding; it maintains that wide-quantification readings really do depend on (covert) syntactic movement, but transparent interpretations of the restrictor do not. The other theory, which we may dub the ``scopal'' account, removes the restrictor from the scope of the modal operator, either by QR (combined with an $\angles{et,t}$ type trace) or by movement of the NP-restrictor by itself.

% In order to adjudicate between these two competing theories, we may want to inquire whether the R-\emph{de re \dash de dicto} distinction exhibits any of the properties that current syntactic theory would take to be diagnostic of movement. This is a very complex enterprise, and the few results to have emerged so far appear to be pointing in different directions.

% We have already mentioned that it is questionable whether NPs that are complements to D can be moved out of their DPs. Even if it is possible, we might expect this movement to be similar to the movement of other predicates, such as APs, VPs, and predicative NPs. Such movements exist, but \dash as discussed by Heycock, Fox, and the sources they cite \dash they typically have no effect on semantic interpretation and appear to be obligatorily reconstructed at LF. The type of NP-movement required by the purely scopal theory of transparent readings would be exceptional in this respect.

% Considerations based on the locality of uncontroversial instances of QR provide another reason to doubt the plausibility of the scopal theory. \citet{may:diss} argued, on the basis of examples like (\nextx), that quantifiers do not take scope out of embedded tensed clauses.

% \pex
% \a Some politician will address every rally in John's district. 
% \a Some politician thinks that he will address every rally in John's district.
% \xe

% While in (\lastx a) the universal quantifier can take scope over the existential quantifier in subject position, this seems impossible in (\lastx b), where the universal quantifier would have to scope out of its finite clause. Therefore, May suggested, we should not attribute the \emph{de re} reading in an example like our (\nextx) to the operation of QR.

% \ex John believes that your abstract will be accepted. \xe

% As we saw above, the standard theory which appeals to non-locally bound world-pronouns does have a way of capturing the \emph{de re} reading of (\lastx) without any movement, so it is consistent with May's suggestion. The purely scopal theory would have to say something more complicated in order to reconcile the facts about (\blastx) and (\lastx). Namely, it might have to posit that DP-movement is finite-clause bound, but NP-movement is not. Or, in the other version, it would have to say that QR can escape finite clauses but only if it leaves a $\angles{et,t}$ type trace.

% Both theories, by the way, have a problem with the fact that May's finite-clause-boundedness does not appear to hold for all quantificational DPs alike. If we look at the behavior of \emph{every}, \emph{no}, and \emph{most}, we indeed can maintain that there is no DP-movement out of tensed complements. For example, (\nextx) could mean that Mary hopes that there won't be any friends of mine that win. Or it could mean (with suitable help from the context) that she hopes that there is nobody who will win among those shaggy people over there (whom I describe as my friends). But it cannot mean merely that there isn't any friend of mine who she hopes will win.

% \ex Mary hopes that no friend of mine will win. \xe

% So (\nextx) has opaque and transparent readings for \emph{no friend of mine}, but no wide-quantification reading where the negative existential determiner \emph{no} takes matrix scope. Compare this with the minimally different infinitival complement structure, which does permit all three kinds of readings.

% \ex Mary expects no friend of mine to win. \xe

% However, indefinite DPs like \emph{a friend of mine}, \emph{two friends of mine} are notoriously much freeer in the scope options for the existential quantifiaction they express. For instance, even the finite clause in (\nextx) seems to be no impediment to a reading that is not only transparent but also wide-quantificational (i.e., it has the existential quantifier over individuals outscoping the universal world-quantifier).

% \ex Mary hopes that a friend of mine will win. \xe

% The peculiar scope-taking behavior of indefinites (as opposed to universal, proportional, and negative quantifiers) has recently been addressed by a number of authors \citep{abusch:1994:indefinites, reinhart:scope:97, winter:choice:97, matthewson:widescope:99, kratzer:pseudo:98}, and there are good prospects for a successful theory that generates even the \emph{wide-Q} transparent readings of indefinites without any recourse to non-local DP-movement. You are encouraged to read these works, but for our current purposes here, all we want to point out is that, with respect to the behavior of indefinites, neither of the two theories we are trying to compare seems to have a special advantage over the other. This is because wide-Q readings result from DP-movement according to \emph{both} theories.

% As we mentioned in the previous chapter, a number of recent papers have been probing the connection between \emph{de dicto} readings and the effects of Binding Condition C applying at LF. These authors have converged on the conclusion that DPs which are read as \emph{de dicto} behave w.r.t. Binding Theory as if they are located below the relevant modal predicate at LF, and DPs that are read as \emph{de re} (i.e., wide-Q, transparent) behave as if they are located above. It is natural to inquire whether the same kind of evidence could also be exploited to determine the LF-location of the NP-part of a DP which is read as narrow-quantificational but restrictor-transparent. If this acted for Condition C purposes as if it were below the attitude verb, it would confirm the standard theory (non-locally bound world-pronouns), whereas if it acted as if it was scoped out, we'd have evidence for the scopal account. \citet{sharvit:howmany:98} constructs some of the relevant examples and reports judgments that actually favor the scopal theory.\footnote{Sharvit's own conclusion, however, is not that her data supports the purely scopal theory.} For example, she observes that (\nextx a) does allow the narrow-Q, transparent-reading indicated in (\nextx b).

% \pex
% \a How many students who like John$_{1}$ does he$_{1}$ think every professor talked to? 
% \a For which $n$ does John think that every professor talked to $n$ people in the set of students who actually like John?
% \xe

% More research is required to corroborate this finding.

% As a final piece of potentially relevant data, consider a contrast in Marathi recently discussed by \citet{bhatt:locality:99}.

% \ex
% \begingl
% \gla\ [ji bai kican madhe ahe]$_{i}$ Ram-la watte ki [[$t_{i}$ [ti bai]$_{i}$ ] kican madhe nahi]//
% \glb \textsc{rel} woman kitchen in is Ram thinks that that woman kitchen in not is//
% \glft `Ram thinks that the woman who is in the kitchen is not in the kitchen' //
% \endgl
% \xe

% \ex
% \begingl
% \gla Ram-la watte ki [ [ji bai kican madhe ahe]$_{i}$ [[$t_{i}$ [ti bai]$_{i}$ ] kican madhe nahi] ]//
% \glb \textsc{rel} woman kitchen in is that woman kitchen in not is//
% \glft `Ram thinks that the woman who is in the kitchen is not in the kitchen'//
% \endgl
% \xe

% The English translation of both examples has two readings: a (plausible) transparent reading, on which Ram thinks of the woman who is actually in the kitchen that she isn't, and an (implausible) \emph{de dicto} reading, on which Ram has the contradictory belief that he would express by saying: ``the woman in the kitchen is not in the kitchen''. The Marathi sentence (\blastx) also allows these two readings, but (\lastx) unambiguously expresses the implausible \emph{de dicto} reading. Bhatt's explanation invokes the assumption that covert movement in Hindi cannot cross a finite clause boundary. In (\blastx), where the correlative clause has moved overtly, it can stay high or else reconstruct at LF, thus yielding either reading. But in (\lastx), where it has failed to move up overtly, it must also stay low at LF, and therefore can only be \emph{de dicto}. What is interesting about this account is that it crucially relies on a scopal account of the transparent-opaque distinction. (Recall that with type-e DPs like definite descriptions, there \emph{is} no additional wide/narrow-Q ambiguity.) If the standard theory with its non-locally bindable world-pronouns were correct, we would not expect the constraint that blocks covert movement in (\lastx) to affect the possibility of a transparent reading.

% In sum, then, the evidence appears to be mixed. Some observations appear to favor the currently standard account, whereas others look like they might confirm the purely scopal account after all. Much more work is needed.

% \section{A Recurring Theme: Historical Overview}

% To recap, the main shape of the phenomenon discussed in this chapter is that the intensional parameter (time, world) with respect to which the predicate restricting a quantifier is interpreted can be distinct from the one that is introduced by the intensional operator that immediately scopes over the quantifier. The crucial cases have the character of a ``scope paradox''. This discovery is one that has been made repeatedly in the history of semantics. It has been made both in the domain of temporal dependencies and in the domain of modality. Here are some of the highlights of that history.\footnote{Some of this history can be found in comments throughout Cresswell's book \citep{cresswell:entities}, which also contains additional references}.

% \begin{enumerate}
  
%   \item The \emph{now}-operator\\[6pt]
%   \citet{prior:now} noticed a semantic problem with the adverb \expression{now}. The main early researchers that addressed the problem were \citet{kamp:now} and \citet{vlach:diss}. A good survey was prepared by \citet{vanbenthem:tenselogic}. Another early reference is \citet{saarinen:backward}. The simplest scope paradox examples looked like this:
  
%   \ex One day all persons now alive will be dead. \xe
  
%   While for this example one could say that \expression{now} is special in always having access to the utterance time, other examples show that an unbounded number of times need to be tracked. It became clear in this work that whether one uses a multitude of indexed \expression{now} and \expression{then}-operators or allows variables over times is a syntactic and not a deep semantic question.
  
%   \medskip\item The \emph{actually}-operator\\[6pt]
%   The modal equivalent of the Prior-Kamp scope paradox sentence is:
  
%   \ex It might have been that everyone actually rich was poor. \xe
  
%   \citet{crossley-humberstone} discuss such examples. Double-indexed systems of modal logic were studied by \citet{segerberg} and \citet{aqvist:1973:modal}. See also work by \citet{lewis:anselm}, \citet{vaninwagen:actuality}, and \citet{hazen:actuality}. Indexed \emph{actually}-operators are discussed by \citet{prior-fine}, \citet{peacocke:necessity}, and \citet{forbes:physicalism,forbes:metaphysics,forbes:possibility}.
  
%   \medskip\item The time of nominal predicates\\[6pt]
%   There is quite a bit of work that argues that freedom in the time-dependency of nominals even occurs when there is no apparent space for temporal operators. Early work includes \citet{enc:diss, enc:L&P:86}. But see also \citet{ejerhed:tense}. More recently Musan's dissertation \citep{musan:diss} is relevant.
  
%   \ex Every fugitive is back in custody. \xe
  
%   \medskip\item Tense in Nominals\\[6pt]
%   There is some syntactic work on tense in nominals, see for example \citet{wiltschko:tenseond}.
  
%   \medskip\item The Fodor-Reading\\[6pt]
%   Examples similar to the ones from Fodor and Bäuerle that we used at the beginning of this chapter are discussed in many places \citep{ioup:specificity77, hellan:scope78, abusch:1994:indefinites, bonomi:transp95, farkas:scope97}. The point that all these authors have made is that the NP-predicate restricting a quantifier may be evaluated in the actual world, even when that quantifier clearly takes scope below a modal predicate.
  
%   Heim \citep{heim:artikel} gives an example like this:
  
%   \ex Every time it could have been the case that the player on the left was on the right instead. \xe
  
%   Here, \expression{the player on the left} must be evaluated with respect to the actual world. But it is inside a tensed clause, which \dash as we saw earlier \dash is usually considered a scope island for quantifiers.
  
%   \medskip\item Explicit World Variables\\[6pt]
%   Systems with explicit world/time variables were introduced by \citet{tichy:approach} and \citet{gallin:intensional}. A system (Ty2) with overt world-variables is used by Groenendijk \amp\ Stokhof in their dissertation on the semantics of questions. See also \citet{zimmermann:diss} on the expressive power of that system.
  
%   \medskip\item Movement\\[6pt]
%   The idea of getting the third reading via some kind of syntactic scoping has not been pursued much. But there is an intriguing idea in a paper by \citet{bricker:pluraldere:89}, cited by \citet[p. 76]{cresswell:entities}. Bricker formalizes a sentence like \expression{Everyone actually rich might have been poor} as follows:
  
%   \ex $\exists X(\forall y(Xy \equiv \mbox{rich } y) \& \diamond\forall y(Xy \rightarrow \mbox{poor }y))$ \xe
  
%   This is apparently meant to be interpreted as `there is a plurality X all of whose members are rich and it might have been the case that all of the members of X are poor'. This certainly looks like somehow a syntactic scoping of the restrictive material inside the universal quantifier out of the scope of the modal operator has occurred.
% \end{enumerate}

% % chapter beyond_emph_de_re_de_dicto_the_third_reading (end)
% % \section{Anaphoric pronouns in modal contexts}
% % 
% % [Before you study this section, please review H\amp K ch. 11 ``E-Type
% % Anaphora.'']
% % 
% % \absatz Let us begin with another example that displays the by now
% % familar \emph{de re}-\emph{de dicto} ambiguity involving a modal verb
% % and a DP in its complement.
% % 
% % \ex I have to return \emph{one of these books}. \xe
% % 
% % This can mean that there is one among these books which I return in
% % every world in which I fulfill my obligations (\emph{de} \emph{re}),
% % and it can mean that in every world in which I fulfill my obligations,
% % I return (a possibly different) one of these books (\emph{de
% % dicto}). Interestingly, only one of these two readings \dash namely the
% % \emph{de re} reading \dash is available when we embed (\lastx) in a text
% % like (\nextx), which has an anaphoric pronoun in the subsequent sentence.
% % 
% % \ex \label{book} I have to return one of these books. But I am not \xe
% % finished with \emph{it}.
% % 
% % Similar disambiguating effects are observed in (\nextx) and (\anextx). By
% % themselves, the first sentences in (\nextx) and (\anextx) are ambiguous
% % between \emph{de} re readings and \emph{de dicto} readings. But only
% % the \emph{de re} readings seem to be consistent with the anaphoric
% % continuations.
% % 
% % \ex \label{cand} Two candidates could get hired. They aren't very \xe
% % optimistic, though.
% % 
% % \ex \label{plumber} Jane wants to marry a plumber. He owns a house. \xe
% % 
% % What explains these judgments?
% % 
% % Recall standard assumptions about pronouns. Setting aside, for the
% % time being, the existence of E-Type pronouns (but we will return to
% % this shortly), a pronoun is a variable (of type $e$). It can be a
% % bound variable or a free variable, and in the latter case it must
% % receive a salient referent from the utterance context. The pronoun in
% % the second sentence of \refx{book} is not in the right environment to
% % be a bound variable, so it has to be analyzed as free. This implies
% % that it refers to an individual, and that this individual is
% % appropriately salient at the point when the pronoun is
% % processed. Since we did not provide any extralinguistic context, it
% % appears to be the utterance of the first sentence in \refx{book} which
% % is responsible for making salient the intended referent of
% % \emph{it}. How does this work?
% % 
% % Well, if the first sentence in \refx{book} is read \emph{de} \emph{re},
% % then it asserts that there is one among these books which the speaker
% % has to return. When the hearer accepts the truth of this assertion
% % (and moreover guesses that there is no more than one such book), then
% % the book which the speaker has to return will be salient to her. So
% % this book is a natural candidate for the reference of the \emph{it},
% % and she will spontaneously understand the second sentence as claiming
% % that the speaker is not yet finished with the book he has to
% % return. This indeed is how the text in \refx{book} appears to be
% % understood.
% % 
% % We have shown, then, that a \emph{de re} disambiguation of the first
% % sentence of \refx{book} provides an appropriate context for the
% % processing of the second sentence with the pronoun in it. So we have
% % explained why a \emph{de} \emph{re} reading is available here; but we
% % have not yet explained why a \emph{de dicto} reading is not. To
% % account for the latter fact, we must argue that the \emph{de dicto}
% % reading is not suitable for singling out a referent for the
% % pronoun. Consider what the \emph{de dicto} reading asserts: it says
% % that in each of the worlds in which the speaker fulfills his
% % obligations, he returns one of these books. In some of those
% % accessible worlds, he returns book A, in others he returns book B, in
% % yet others he returns book C. (Suppose that the plurality referred to
% % by \emph{these books} consists of just A, B, and C.) There is then no
% % one book among these three that is especially singled out as the one
% % that supports the truth of the \emph{de dicto} assertion. Therefore,
% % we submit, the use of the free pronoun \emph{it} is not felicitous
% % when it follows the \emph{de dicto} assertion. Since the hearer takes
% % for granted that the text as a whole is a felicitous utterance, she
% % spontaneously disambiguates the first sentence in favor of its
% % \emph{de re} reading.
% % 
% % The general prediction which emerges from this discussion is the
% % following: When a DP in the complement of a modal operator is intended
% % as the antecedent of a pronoun outside that operator's scope, then
% % this DP must be read \emph{de re} \dash more precisely, it must be
% % construed with scope over the modal operator. The situation is exactly
% % analogous when we look at examples which involve an ordinary
% % quantifier over individuals instead of the modal operator. Compare
% % \refx{book} to (\nextx).
% % 
% % \ex \label{rev} One of the reviewers has read every abstract. I have \xe
% % talked to her.
% % 
% % In isolation, the first sentence of \refx{rev} is scopally ambiguous,
% % and permits, in particular, a reading which is true if different
% % abstracts are read by different reviewers. But this reading is not
% % available with the continuation in \refx{rev}. The need to identify a
% % salient referent for the pronoun \emph{her} forces us to read the
% % preceding sentence with wide scope for the subject. By adopting a
% % quantificational approach to modal predicates and a scopal account of
% % \emph{de re}-\emph{de dicto} ambiguities, we are led to perceive the
% % semantic structures of \refx{rev} and \refx{book} as completely parallel
% % and to expect analogous disambiguating effects from the presence of a
% % referential anaphoric pronoun. And this is just what we have found.
% % 
% % A closer look at the data reveals that an anaphoric pronoun does not
% % always force a \emph{de re} reading of its antecedent. Consider the
% % following variants of \refx{book}, \refx{cand}, and \refx{plumber}.
% % 
% % \ex \label{book2} I have to return one of these books. I am supposed \xe
% % to drop it off tomorrow.
% % 
% % \ex \label{cand2} Two candidates could get hired. They could get \xe
% % half-time positions.
% % 
% % \ex \label{plumber2} Jane wants to marry a plumber. He has to own a \xe
% % house.
% % 
% % In these examples, \emph{de dicto} readings of indefinite DPs in the
% % initial sentences are intuitively available, even when the pronouns
% % take these DPs as antecedents. How come? One difference between these
% % examples and the ones in \refx{book}-\refx{plumber} is that here we have
% % another modal operator in the second sentence (the sentence containing
% % the pronoun). We will see that this is significant.
% % 
% % The account that we will propose for the availability of \emph{de
% % dicto} readings in \refx{book2}-\refx{plumber2} is an application (or
% % extension) of the E-Type analysis of pronouns discussed in ch. 11 of
% % H\amp K. The informal idea behind the E-Type analysis, as you recall,
% % was that certain pronouns are interpreted as if they were definite
% % descriptions that contain a bound variable. For the pronouns in
% % \refx{book2}-\refx{plumber2}, we will propose, in a nutshell, that they
% % are like definite descriptions which contain a bound \emph{world}
% % variable. To see the idea, consider the following rendition of the
% % truth-conditions of \refx{book2}.
% % 
% % \ex For every accessible world $w$: in $w$, I return one of these \xe
% % books. And moreover, for every accessible world $w$: in $w$, I drop
% % \emph{the book I return in $w$} off tomorrow.
% % 
% % The underlined definite in (\lastx) corresponds to the pronoun \emph{it}
% % in \refx{book2}. It picks out a unique book for each (accessible) world
% % $w$, but not necessarily the same book in different worlds. Suppose
% % again that \emph{these books} refers to the plurality comprising books
% % A, B, and C. In the first sentence of \refx{book2}, read \emph{de
% % dicto}, I claim that every world in which I fulfill my obligations is
% % a world in which I return A, B, or C. In some of these worlds I return
% % A, in others B, in yet others C. The second sentence of \refx{book2}
% % quantifies over the same set of worlds: \emph{supposed-to} here also
% % ranges over the worlds in which I fulfill my obligations. The
% % \emph{supposed-to} sentence now says that in each such world, I drop
% % off tomorrow the book that I return in that world. So in those worlds
% % where I return A, I drop A off tomorrow; in those where I return B, I
% % drop B off tomorrow; and in those where I return C, I drop C off
% % tomorrow. Indeed, these seem to be the truth-conditions which we
% % actually get for \refx{book2} when we read the first sentence \emph{de
% % dicto} and the second with the \emph{it} anaphoric to \emph{one of
% % these books}.
% % 
% % The technical implementation of this account is a straightforward
% % combination of the material from chapter 11 with our current treatment
% % of modal predicates. Recall that in ch. 11 we decided to represent
% % E-Type pronouns at LF as consisting of a covert definite article
% % followed by a predicate made up of two variables: a free
% % relation-variable (type \angles{e,et}) and a bound pronoun of type
% % $e$. In the cases that we are considering here now, the bound
% % ``pronoun'' should be of type $s$ (taking worlds as values) and the
% % free relation-variable correspondingly of type \angles{s,et}.
% % 
% % Suppose then that the LF of the second sentence of \refx{book2} looks
% % like this:
% % 
% % \exi. [ $w$-{\scshape pro} $\lambda_2$ [ [ supposed $t_{s,2}$ $R$ ] [
% % $w$-{\scshape pro} $\lambda_1$ [VP I [ [ drop-off-tomorrow $t_{s,1}$ ]
% % [ the $r_{\angles{s,et}}\ w$-pro$_1$ ]]]]]]
% % 
% % The subtree representing the pronoun \emph{it} here contains the
% % definite article \emph{the} and a free variable $r$ of type
% % \angles{s,et}. $r$ must receive a value from the context, and for the
% % intended reading of the example, we assume that its value is the
% % function which maps each world $w$ to the set of books that I return
% % in $w$.\footnote{Notice that this function is defined for all possible
% % worlds, not just those which are accessible under (the intended value
% % for) $R$, i.e. those where I fulfill my obligations. It is therefore
% % not guaranteed that when applied to an arbitrary argument, this
% % function will pick out a singleton set containing at least and at most
% % one book. However, if we take the first sentence of \refx{book2} to be
% % true (i.e., we take it to be true that in each $R$-accessible world I
% % return one of A, B, C), and if moreover we disregard any
% % $R$-accessible worlds in which I return more than one book, then we
% % can be sure that $r$ picks out a set of exactly one book in every
% % $R$-accessible world (under consideration).} (We take it that this
% % function has become salient to the hearer as a result of processing
% % the first sentence of \refx{book2}.) Since, $r$ requires as its first
% % argument a world, we generate a world-pronominal as its sister. This
% % pronominal can be coindexed with the abstractor over worlds introduced
% % by the movement of the vacuous w-PRO in the world-argument position of
% % the verb \emph{drop off}.
% % 
% % In%\mycite{partee:1972}\mycite{roberts:1989}\mycite{poesio-zucchi:1992}
% % this section, we have tried to show that the scopal treatment of
% % \emph{de re}-\emph{de dicto} ambiguities interacts with an
% % independently motivated theory of pronoun interpretation, and to
% % confirm at least some of the resulting empirical predictions about the
% % interpretations of anaphoric pronouns and their antecedents in
% % modalized sentences. Of course, we have only considered a very small
% % set of data, and the full picture of the facts is more complicated.
% % 
% % \begin{exercise}
% % 
% %   Discuss the following passage from Montague's \cite{montague:1973}
% % famous paper ``The Proper Treatment of Quantification in Ordinary
% % English'' (PTQ):
% % 
% %   \begin{quote}
% % 
% %     \dots The next example indicates the necessity of allowing verb
% % phrases as well as sentences to be conjoined and quantified. Without
% % such provisions the sentence \emph{John wishes to find a unicorn and
% % eat it} would (unacceptably, as several linguists have pointed out in
% % connection with parallel examples) have only a ``referential''
% % reading, that is, one that entails that there are unicorns. [\dots]
% % The next example is somewhat simpler, in that it does not involve
% % conjoining or quantifying verb phrases; but it also illustrates the
% % possibility of a nonreferential reading in the presence of a pronoun.
% % 
% %     \ex Mary believes that John finds a unicorn and he eats it \xe
% % 
% %     [\dots]
% % 
% %     \medskip On the other hand, in each of the following examples only
% % one reading is possible, and that [is] the referential:
% % 
% %     \ex John seeks a unicorn and Mary seeks it \xe
% % 
% %     \ex \label{uni} John tries to find a unicorn and wishes to eat it \xe
% % 
% %     [\dots]
% % 
% %     \medskip This is, according to my intuitions (and, if I guess
% % correctly from remarks in Partee [1970], those of Barbara Partee as
% % well), as it should be; but David Kaplan would differ, at least as to
% % \refx{uni}. Let him, however, and those who might sympathize with him
% % consider the following variant of \refx{uni} and attempt to make
% % nonreferential sense of it:
% % 
% %     \ex \label{uni2} John wishes to find a unicorn and tries to eat \xe
% % it
% % 
% %   \end{quote}
% % 
% %   \noindent There are (at least) two points worth scrutinizing here:
% % First, Montague assumes that his first example (\emph{John wishes to
% % find a unicorn and eat it}) is appropriately treated by giving \emph{a
% % unicorn} scope over the coordinate VP and letting it bind \emph{it} as
% % an ordinary bound variable pronoun. This assumption might be
% % problematic in light of analogous examples such as \emph{John wants to
% % buy just one bottle of wine and serve it with the main course}.
% % 
% %   Second, there is the claim that \refx{uni} has only a ``referential''
% % reading, qualified by a reference to diverging judgments. (Partee
% % 1970, incidently, seems to side with Kaplan's rather than Montague's
% % judgment; see her remark on her example (52), p. 373: \emph{John was
% % trying to catch a fish}. \emph{He wanted to eat it for supper}.) What
% % is interesting here is that Montague switches to \refx{uni2} to obtain
% % a clearer judgment. He seems to assume that any reasonable theory that
% % predicted a ``nonreferential'' reading for \refx{uni} would have to do
% % the same for \refx{uni2}. Is this assumption justified? In particular,
% % if the ``nonreferential'' reading of \refx{uni} were to be given an
% % E-Type analysis, might it be feasible to spell out the semantics of
% % \emph{wish} and \emph{try} in such a way that an analogous reading
% % would not automatically arise for \refx{uni2}? \eex
% % 
% % \end{exercise}

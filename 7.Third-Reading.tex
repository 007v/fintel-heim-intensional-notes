\excnt=1
\chapter{The Third Reading}\label{cha:the_third_reading} 

\chapterprecishere{In this chapter, we will see that quantificational noun
  phrases in the scope of a modal operator can receive a reading where their
  restrictive predicate is not interpreted in the worlds introduced by the modal
  operator (which is what happens in specific readings as well) while \emph{at
    the same time} their quantificational force takes scope below the modal
  operator (which is what happens in non-specific readings as well). This
  seemingly paradoxical situation might force whole-sale revisions to our
  architecture. We discuss the standard solution (which involves supplying
  predicates with world-arguments).
}

\minitoc

\section{A Problem}

Janet Dean Fodor discussed examples like (\nextx) in her dissertation
(\citeyear{fodor-diss}).

\ex Mary wanted to buy a hat just like mine. \xe
%
Fodor observes that (\lastx) has three readings, which she labels ``specific
transparent,'' ``non-specific transparent,'' and ``non-specific opaque.''

\begin{enumerate}[(i)] 
\item On the ``specific transparent'' reading, the sentence says that there is a
  particular hat which is just like mine such that Mary has a desire to buy it.
  Say, I am walking along Newbury Street with Mary. Mary sees a hat in a display
  window and wants to buy \emph{it}. She tells me so. I don't reveal that I have
  one just like it. But later I tell \emph{you} by uttering (\lastx).
\item On the ``non-specific opaque'' reading, the sentence says that Mary's
  desire was to buy some hat or other which fulfills the description that it is
  just like mine. She is a copycat.
\item On the ``non-specific transparent'' reading, finally, the sentence will be
  true, e.g., in the following situation: Mary's desire is to buy some hat or
  other, and the only important thing is that it be a Red Sox cap. Unbeknownst
  to her, my hat is one of those as well.
\end{enumerate}
%
The existence of three different readings appears to be problematic for the
scopal account of specific/non-specific ambiguities that we have been assuming.
It seems that our analysis allows just two semantically distinct types of LFs:
Either the DP \emph{a hat just like mine} takes scope below \emph{want}, as in
(\nextx), or it takes scope above \emph{want}, as in (\anextx).

\ex Mary wanted [ [a hat-just-like-mine]$_{1}$ [ \textsc{pro} to buy $t_{1}$ ]] \xe

\ex\ [a hat-just-like-mine]$_{1}$ [ Mary wanted [ \textsc{pro} to buy $t_{1}$ ]] \xe

In the system we have developed so far, (\blastx) says that in every world $w'$
in which Mary gets what she wants, there is something that she buys in $w'$
that's a hat in $w'$ and like my hat in $w'$. This is Fodor's ``non-specific
opaque'' reading. (\lastx), on the other hand, says that there is some thing $x$
which is a hat in the actual world and like my hat in the actual world, and Mary
buys $x$ in every one of her desire worlds. That is Fodor's ``specific
transparent.'' But what about the ``non-specific transparent'' reading? To
obtain this reading, it seems that we would have to evaluate the predicate
\emph{hat just like mine} in the actual world, so as to obtain its actual
extension (in the scenario we have sketched, the set of all Red Sox caps). But
the existential quantifier expressed by the indefinite article in the
\emph{hat}-DP should not take scope over the modal operator \emph{want}, but
below it, so that we can account for the fact that in different desire-worlds of
Mary's, she buys possibly different hats.

There is a tension here: one aspect of the truth-conditions of this reading
suggests that the DP \emph{a hat just like mine} should be \emph{outside} of the
scope of \emph{want}, but another aspect of these truth-conditions compels us to
place it \emph{inside} the scope of \emph{want}. We can't have it both ways, it
would seem, which is why this has been called a ``scope paradox''

Another example of this sort, due to \citet{bauerle:1983:nps}, is (\nextx):

\ex Georg believes that a woman from Stuttgart loves every member of the VfB team. \xe
%
Bäuerle describes the following scenario: Georg has seen a group of men on the
bus. This group happens to be the VfB team (Stuttgart's soccer team), but Georg
does not know this. Georg also believes (Bäuerle doesn't spell out on what
grounds) that there is some woman from Stuttgart who loves every one of these
men. There is no particular woman of whom he believes that, so there are
different such women in his different belief-worlds. Bäuerle notes that (\lastx)
can be understood as true in this scenario. But there is a problem in finding an
appropriate LF that will predict its truth here. First, since there are
different women in different belief-worlds of Georg's, the existential
quantifier \emph{a woman from Stuttgart} must be inside the scope of
\emph{believe}. Second, since (in each belief world) there aren't different
women that love each of the men, but one that loves them all, the \emph{a}-DP
should take scope over the \emph{every}-DP. If the \emph{every}-DP is in the
scope of the \emph{a}-DP, and the \emph{a}-DP is in the scope of \emph{believe},
then it follows that the \emph{every}-DP is in the scope of \emph{believe}. But
on the other hand, if we want to capture the fact that the men in question need
not be VfB-members in Georg's belief-worlds, the predicate \emph{member of the
  VfB team} needs to be outside of the scope of \emph{believe}. Again, we have a
``scope paradox''.

Before we turn to possible solutions for this problem, let's have one
more example:

\ex Mary hopes that a friend of mine will win the race. \xe
%
This again seems to have three readings. In Fodor's terminology, the DP \emph{a
  friend of mine} can be ``non-specific opaque,'' in which case (\lastx) is true
iff in every world where Mary's hopes come true, there is somebody who is my
friend and wins. It can also have a ``specific transparent'' reading: Mary wants
John to win, she doesn't know John is my friend, but I can still report her hope
as in (\lastx). But there is a third option, the ``non-specific transparent''
reading. To bring out this rather exotic reading, imagine this: Mary looks at
the ten contestants and says \emph{I hope one of the three on the right wins -
  they are so shaggy - I like shaggy people}. She doesn't know that those are my
friends. But I could still report her hope as in (\lastx).

\section{The Standard Solution: Overt World Variables}

The scope paradoxes we have encountered can be traced back to a basic design
feature of our system of intensional semantics: the relevant ``evaluation
world'' for each predicate in a sentence is strictly determined by its
LF-position. All predicates that occur in the (immediate) scope of the same
modal operator must be evaluated in the same possible worlds. E.g. if the scope
of \emph{want} consists of the clause \emph{a friend of mine (to) win}, then
every desire-world $w'$ will be required to contain an individual that wins in
$w'$ and is also my friend \emph{in $w'$}. If we want to quantify over
individuals that are my friends in the actual world (and not necessarily in all
the subject's desire worlds), we have no choice but to place \emph{friend of
  mine} outside of the scope of \emph{want}. And if we want to accomplish this
by means of QR, we must move the entire DP \emph{a friend of mine}.

Not every kind of intensional semantics constrains our options in this way. One
way to visualize what we might want is to write down an LF that looks promising:

\ex Mary wanted$_{w_0} [ \lambda w' [$ a hat-just-like-mine $_{w_0}] \lambda x_1 [$ \textsc{pro} to buy$_{w'} x_{1} ]]$ \xe

We have annotated each predicate with the world in which we wish to evaluate it.
$w_0$ is the evaluation world for the entire sentence and it is the world in
which we evaluate the predicates \expression{want} and
\expression{hat-just-like-mine}. The embedded sentence contributes a function
from worlds to truth-values and we insert an explicit $\lambda$-operator binding
the world where the predicate \expression{buy} is evaluated. The crucial aspect
of (\lastx) is that the world in which \expression{hat-just-like-mine} is
evaluated is the matrix evaluation world and not the same world in which its
clause-mate predicate \expression{buy} is evaluated. This LF thus looks like it
might faithfully capture Fodor's third reading.

Logical forms with overt world variables such as (\lastx) are in fact the
standard solution to the problem presented by the third reading. Let us spell
out some of the technicalities. Later, we will consider a couple of
alternatives.

We return to the basic system used in Heim \amp\ Kratzer up to chapter 11. The
interpretation function is relativized only to an assignment function, not to
any other evaluation parameters such as a world, a time, or an index. The
semantic rules are Functional Application, Predicate Abstraction, and Predicate
Modification, in their formulations from the earlier part of H\amp K. There is
no rule of Intensional Functional Application. The only ingredient of
intensional semantics that we do retain is the expanded type system and
ontology. We have a third basic type besides $e$ and $t$, the type $s$. $D_s$ is
the set of all indices, for now possible worlds (later: world-time pairs).

There are a number of innovations in the lexicon and in the syntax. As for the
lexicon, the main change concerns the treatment of predicates (verbs, nouns,
adjectives). They now all get an additional argument, of type $s$.\marginnote{The
  decision to make the world-argument the predicate's first (lowest) argument is
  arbitrary, and nothing hinges on it. For all we know, it could be the highest
  argument, or somewhere in between.}

\pex\label{124}
\a $\sv{\mbox{smart}} = \lambda w\in D_{s}.\ \lambda x\in D_{e}.\ x$ is smart in $w$ 
\a $\sv{\mbox{likes}} = \lambda w\in D_{s}.\ \lambda x\in D_{e}.\ \lambda y\in D_{e}$. $y$ likes $x$ in $w$ 
\a $\sv{\mbox{teacher}} = \lambda w\in D_{s}.\ \lambda x\in D_{e}$. $x$ is a teacher in $w$ 
\a $\sv{\mbox{friend}} = \lambda w\in D_{s}.\ \lambda x\in D_{e}.\ \lambda y\in D_{e}$. $y$ is $x$'s friend in $w$
\xe

This also applies to attitude predicates, modals, and tenses. We illustrate with
\emph{believe} and \emph{must}:

\pex
\a $\sv{\mbox{believe}} = \lambda w\in D_{s}.\ \lambda p\in D_{\angles{s,t}}.\ \lambda x\in D.$ \\
\null\hfill$\forall w'$ [$w'$ conforms to what $x$ believes in $w \rightarrow p(w') = 1$] 
\a $\sv{\mbox{must}} = \lambda w\in D_{s}.\ \lambda R\in D_{\angles{s,st}}.\ \lambda p\in D_{\angles{s,t}}$.\\
\null\hfill$\forall w'\ [R(w)(w') = 1 \rightarrow p(w') = 1]$
\xe
%
Note that predicates (ordinary ones and modal ones), like the ones in \refx{124}
and (\lastx) now have as their semantic values what used to be their
\emph{intensions}.

There is no change to the entries of proper names, determiners, or
truth-functional connectives; these keep their purely extensional (``s-free'')
types and meanings:

\pex
\a $\sv{\mbox{Ann}} = $Ann 
\a $\sv{\mbox{and}} = \lambda u\in D_{t}.\ [ \lambda v\in D_{t}.\ u=v=1]$ 
\a $\sv{\mbox{the}} = \lambda f\in D_{\angles{e,t}}\co \exists !x.\ f(x) = 1.$ the $y$ such that $f(y) = 1$. 
\a $\sv{\mbox{every}} = \lambda f\in D_{\angles{e,t}}.\ \lambda g\in D_{\angles{e,t}}.\ \forall x [ f(x) = 1 \rightarrow g(x) = 1]$
\xe

So, let's start analyzing a simple sentence.

\ex\ [$_{VP}$ John leaves] \xe
%
The verb's type is $\angles{s,et}$, so it's looking for a sister node which
denotes a \emph{world}. \emph{John}, which denotes an individual, is not a
suitable argument. 

We get out of this problem by adding a couple of items to our lexicon, which are
abstract (unpronounced) morphemes. One is a series of pronouns of type $s$
(``index pronouns'' or, for now, ``world pronouns''). In this chapter, we will
write them as $w_n$, with a numerical subscript $n$, or even as $w, w', w''$.
(Later, we sometimes might write them as \emph{pro}$_n$ and rely on context to
make clear we are not referring to an individual.) Their semantics is what you
expect: they get values from the assignment function.

We will stipulate that a complete (matrix) sentence must not contain any free
variables of type $s$ and must receive a denotation of type
\type{s,t}.\footnote{In the 2016 edition of this class, Suzana Fong noted that
  this stipulation is prima facie less appealing than the alternative assumption
  that type-$s$ pronouns are exactly like type-$e$ pronoun in every respect,
  including the ability to remain free and get values from a contextually
  supplied assignment. Irene tried to sketch some principled reason why it might
  not be possible to refer to a specific world other than the world one is in.
  But as Mitya Privoznov pointed out, a similar idea is not plausible for times,
  given the existence of temporal deictics like \emph{then}. So at best there
  might be a principled reason why the world-coordinate of a free index-pronoun
  would always have to be $w_u$. Irene had to concede therefore that the ban
  against free index-pronouns was just a stipulation. We want to think more
  about (a) whether we really need it, and (b) if we do, what might explain it.}
This means that we need binders of world pronouns. Many proposals in this line
of thought help themselves to freely inserted covert binders. We will follow
H\amp K in not doing that. Instead we posit one more lexical item, analogous to
the covert vacuous operator \emph{PRO} of type $e$ in H\amp K (pp.227-228): a
semantically vacuous operator, \emph{OP}, which moves and leaves a trace of type
$s$. Its syntactic properties are such that it must end up in C or right below a
functional head in the ``clausal spine'' between C and V, and it must get there
by a very short movement, a kind of ``head movement''. We are leaving this
rather vague.

So, our sentence \emph{John leaves} contains \emph{OP}, generated as the first
sister of the verb and then moved to the ``top'' of the sentence:

\ex \emph{OP} 1 [ John [ leaves $t_1$ ]] \xe
%
Our system generates the following denotation for (\lastx): ``$\lambda w_s.$
John leaves in $w$'', a proposition. We rewrite the definition of truth/falsity
of an utterance as follows:

\ex An utterance of a sentence (=LF) $\phi$ in world $w$ is true iff
$\sv{\phi}($w$) = 1$. \xe
%
So, if we utter our sentence in this world (call it $w_@$), then the utterance
was true iff John leaves in $w_@$.

Now, we have to look at more complex sentences. First, a simple case of
embedding. The sentence is \expression{John wants to leave}, which now has an LF
like this:

\ex $[\ \emph{OP}\ 1\ [ \mbox{ John } [ \mbox{ wants } t_{1}\ [\ \emph{OP}\ 2\ [
\mbox{ \textsc{pro}(= John) } [ \mbox{ leave } t_{2}\ ]]]]]]$ \xe
 
\begin{exercise}
	Calculate the semantic value of (\lastx). \eex
\end{exercise}

Next, look at an example involving a complex subject, such as \expression{the
  teacher left}. The verb will need a world argument as before. The noun
\expression{teacher} will likewise need one, so that \expression{the} can get
the required argument of type $\angles{e,t}$ (not $\angles{s,et}$!). Now, our
system makes an interesting prediction: one of the world arguments has to be
\emph{OP} and one of them has to be a pronoun. (Why?) We have free choice, it
appears, as to which predicate gets which kind of world argument. Let's assume,
for now, that we insert \emph{OP} as the sister of the verb and a world pronoun
$w_?$ as the sister of the noun. Since we have stipulated that a complete
sentence cannot contain any free world pronouns, the operator and the pronoun
have to be co-indexed. So, after \emph{OP}-movement, we will have this LF:

\ex \emph{OP} 1 [ [ The [ teacher $w_1$ ]] [ left $t_1$ ]] \xe
%
This will denote the correct proposition (true of a world $w$ iff the unique
individual who is a teacher in $w$ left in $w$).

Now comes the payoff. Consider what happens when the sentence contains both a
modal operator and a complex DP in its complement.

\ex \label{fom} Mary wants a friend of mine to win. \xe
%
There are now three predicates that need world arguments. Furthermore, since
\emph{want} needs a proposition as its second argument (after its world
argument), there needs to be an \emph{OP} on top of the embedded clause. There
also needs to be an \emph{OP} on top of the matrix clause. As before, \emph{a
  friend of mine} can stay in the embedded clause or QR into the matrix clause.
When it moves into the matrix clause, the only way to not leave its world
argument free is to co-index it with the matrix \emph{OP}. But when it stays
below, we can choose to co-index it with either \emph{OP}, which is how we
generate two non-specific readings, one opaque and one transparent. Here are the
three LFs (to make the structures more readable, we leave off most of the
bracketing and start writing the world arguments as subscripts to the
predicates):\marginnote{So, instead of writing $t_1$ for a trace of type $s$
  that serves as the first argument of \emph{leaves}, say, we write ``leaves$_{w_1}$''}

\pex
\a non-specific opaque:\\
\emph{OP} 1 Mary wants$_{w_1}$ [ \emph{OP} 2 a friend-of-mine$_{w_2}$ leave$_{w_2}$ ]
\a specific transparent:\\
\emph{OP} 1 a friend-of-mine$_{w_1}$ 3 Mary wants$_{w_1}$ [ \emph{OP} 2 $t_3$ leave$_{w_2}$
]
\a non-specific transparent:\\
\emph{OP} 1 Mary wants$_{w_1}$ [ \emph{OP} 2 a friend-of-mine$_{w_1}$ leave$_{w_2}$ ]
\xe
%
Notice that the third reading is minimally different from the first reading: all
that happened is the choice to co-index the world argument of \emph{friend of
  mine} with the matrix \emph{OP}.

In this new framework, then, we have a way of resolving the apparent ``scope
paradoxes'' and of acknowledging Fodor's point that there are two separate
distinctions to be made when DPs interact with modal operators. First, there is
the scopal relation between the DP and the operator; the DP may take wider scope
(Fodor's ``specific'' reading) or narrower scope (``non-specific'' reading) than
the operator. Second, there is the choice of binder for the world-argument of
the DP's restricting predicate; this may be cobound with the world-argument of
the embedded predicate (Fodor ``opaque'') or with the modal operator's own
world-argument (``transparent''). So the transparent/opaque distinction in the
sense of Fodor is not \emph{per se} a distinction of scope; but it has a
principled connection with scope in one direction: Unless the DP is within the
modal operator's scope, the opaque option (= co-binding the world-pronoun with
the embedded predicate's world-argument) is in principle unavailable. (Hence
``specific'' implies ``transparent'', and ``opaque'' implies ``non-specific''.)
But there is no implication in the other direction: if the DP has narrow scope
w.r.t. to the modal operator, either the local or the long-distance binding
option for its world-pronoun is in principle available. Hence ``non-specific''
readings may be either ``transparent'' or ``opaque''.

\begin{exercise}\label{yourabstract}
	
	For DPs with extensions of type $e$ (specifically, DPs headed by the definite
  article), there is a truth-conditionally manifest transparent/opaque
  distinction, but no truth-conditionally detectable specific/non-specific
  distinction. In other words, if we construct LFs analogous to (\lastx)[a-c]
  above for an example with a definite DP, we can always prove that the first
  option (wide scope DP) and the third option (narrow scope DP with distantly
  bound world-pronoun) denote identical propositions. In this exercise, you are
  asked to show this for the example in (\nextx).
	
	\ex John believes that your abstract will be accepted. \eex \xe

\end{exercise}

\section{The third reading with conditionals and modals}
\label{sec:third-conditional-modal}

So, far our examples of the third reading have all been with attitude predicates
but the phenomenon can also be observed in conditionals and with modals. A
famous example is due to \cite{abusch:1994:indefinites}:

\ex
Things would be different if every senator had grown up to be a rancher instead. 
\xe
%
What makes conditionals different is that the \emph{if}-clause is a scope island
for quantifiers so that \emph{every senator} cannot QR scope out of the
\emph{if}-clause in (\lastx). But the question of whether its predicate,
\emph{senator}, is interpreted in the matrix evaluation world (``transparent'')
or in the worlds that \emph{if} takes us to (``opaque'') remains open. Abusch's
example is constructed to heavily favor the transparent reading.

\cite{percus-2000-constraints} provides a clever minimal pair that shows the
expected ambiguity:

\pex
\a If every semanticist owned a villa in Tuscany, there would be no field
at all.
\a If I were a syntactician and if every semanticist owned a villa in Tuscany, I
would be quite envious.
\xe

We can also see the ambiguities at work in modal sentences:

\pex
\a It could have been that everyone inside was outside.
\a Everyone inside is permitted to be outside.
\xe
%
These latter examples are from \cite{yalcin-2015-epistemic-de-re}, who proceeds
to discuss the very puzzling fact that transparent readings do not seem to be
available in certain ``epistemic'' contexts, neither with indicative
conditionals nor modals.

\section{Binding Theory for World Variables}

One could in principle imagine some indexings of our LFs that we have not
considered so far. In a system (unlike ours) where one freely inserts ``$\lambda
w$'' operators on top of every clause, one could generate the following LF,
which indexes the predicate of the complement clause to the matrix
$\lambda$-operator rather than to the one on top of its own clause.

\ex $\lambda w_{0} \mbox{ John wants}_{w_{0}} [ \lambda w_{1} \mbox{ \textsc{pro} } \mbox{leave}_{w_{0}} ]$ \xe

Of course, the resulting semantics would be pathological: what John would be
claimed to stand in the wanting relation to is a set of worlds that is either
the entire set $W$ of possible worlds (if the evaluation world is one in which
John leaves) or the empty set (if the evaluation world is one in which John
doesn't leave). Clearly, the sentence has no such meaning. Would we need to
restrict our system to not generate such an LF? Perhaps not, if the meaning is
so absurd that the LF would be filtered out by some overarching rules
distinguishing sense from nonsense. Nevertheless, it is gratifying to note that
this kind of LF is unavailable in our system: the only place where the lower
world-binder \emph{OP} could originate is as a sister to \emph{leave} and moving
it to the lower CP (where we need a proposition to feed to \emph{want}) would by
necessity make it bind the world argument of \emph{leave}.

But there are real problems when we look at more complex examples. Here is one
discussed by Percus in important work \parencite{percus-2000-constraints}:

\ex Mary thinks that my brother is Canadian. \xe
%
Since the subject of the lower clause is a type $e$ expression, we expect at least
two readings: opaque and transparent, cf. Exercise \ref{yourabstract}. The two
LFs are as follows:

\pex
\a opaque\\
$\emph{OP } 0 \mbox{ Mary thinks}_{w_{0}} [ \mbox{ (that) } \emph{OP } 1 \mbox{ my brother}_{w_{1}} \mbox{ (is) Canadian}_{w_{1}} ]$
\a transparent\\
$\emph{OP } 0 \mbox{ Mary thinks}_{w_{0}} [ \mbox{ (that) } \emph{OP } 1 \mbox{ my brother}_{w_{0}} \mbox{ (is) Canadian}_{w_{1}} ]$
\xe

But as Percus points out, there is another indexing that might be generated:

\ex $\emph{OP } 0 \mbox{ Mary thinks}_{w_{0}} [ \mbox{ (that) } \emph{OP } 1 \mbox{ my brother}_{w_{1}} \mbox{ (is) Canadian}_{w_{0}} ]$ \xe
%
In (\lastx), we have co-indexed the main predicate of the lower clause with the
matrix $\lambda$-operator and co-indexed the nominal predicate
\expression{brother} with the embedded $\lambda$-operator. That is, in
comparison with the transparent reading in (\blastx b), we have just switched
around the indices on the two predicates in the lower clause.

Note that this LF will not lead to a pathological reading. So, is the predicted
reading one that the sentence actually has? No. For the transparent reading, we
can easily convince ourselves that the sentence does have that reading. Here is
Percus' scenario: ``My brother's name is Allon. Suppose Mary thinks Allon is not
my brother but she also thinks that Allon is Canadian.'' In such a scenario, our
sentence can be judged as true, as predicted if it can have the LF in (\blastx
b). But when we try to find evidence that (\lastx) is a possible LF for our
sentence, we fail. Here is Percus:

\begin{quote}
  If the sentence permitted a structure with this indexing, we would take the
  sentence to be true whenever there is some \emph{actual} Canadian who
  \emph{Mary thinks} is my brother \dash even when this person is not my brother
  in actuality, and \emph{even when Mary mistakenly thinks that he is not
    Canadian}. For instance, we would take the sentence to be true when Mary
  thinks that Pierre (the Canadian) is my brother and naturally concludes \dash
  since she knows that \emph{I} am American \dash that Pierre too is American.
  But in fact we judge the sentence to be \emph{false} on this scenario, and so
  there must be something that makes the indexing in (\lastx) impossible.
\end{quote}
%
Percus then proposes the following descriptive generalization:

\ex \textsc{Generalization X}: The situation pronoun that a verb selects for
must be coindexed with the nearest $\lambda$ above it.\marginnote{Percus works
  with situation pronouns rather than world pronouns, an immaterial difference
  for our purposes here.} \xe
%
We expect that there will need to be a lot of work done to understand the deeper
sources of this generalization. But note that we could implement the constraint
in our system by brute force: the \emph{OP} operator can only be generated as
the sister of a main predicate, not as the sister of a predicate inside an
argument nominal.

\section{Excursus: Semantic reconstruction revisited}

Let us look back at the account of non-specific readings of raised subjects that
we sketched earlier in Section \ref{sem}. We showed that you can derive such
readings by positing a high type trace for the subject raising, a trace of type
$\angles{s,\angles{et,t}}$. Before the lower predicate can combine with the
trace, the semantic value of the trace has to be extensionalized by being
applied to the lower evaluation world (done via the EFA composition principle).
Upstairs the raised subject has to be combined with the $\lambda$-abstract
(which will be of type $\angles{\angles{s,\angles{et,t}},t}$) via its intension.

We then saw data suggesting that syntactic reconstruction is actually what is
going on. This, of course, raises the question of why semantic reconstruction is
unavailable (otherwise we wouldn't expect the data that we observed).

In this excursus, we will briefly consider whether our new framework has
something to say about this issue. Let's figure out what we would have to do in
the new framework to replicate the account in the section on semantics
reconstruction.

Downstairs, we would have a trace of type $\angles{s,\angles{et,t}}$. To
calculate its extension, we do not need recourse to a special composition
principle, but can simply give it a world-argument (co-indexed with the
abstractor resulting from the movement of the $w$-\emph{OP} in the argument
position of the lower verb).

Now, what has to happen upstairs? Well, there we need the subject to be of type
$\angles{s,\angles{et,t}}$, the same type as the trace, to make sure that its
semantics will enter the truth-conditions downstairs. But how can we do this?

We need the DP \emph{somebody from New York} to have as its semantic value an
intension, the function from any world to the existential quantifier over
individuals who are people from New York in that world. This is actually hard to
do in our system. It \emph{would} be possible if (i) the predicate(s) inside the
DP received $w$-\textsc{pro} as their argument, and if (ii) that
$w$-\textsc{pro} were allowed to moved to adjoin to the DP. If we manage to rule
out at least one of the two preconditions on principled grounds, we would have
derived the impossibility of semantic reconstruction as a way of getting
non-specific readings of raised subjects.

\begin{enumerate}[(i)] 
\item may be ruled out by the Binding Theory for world pronominals, when it gets
  developed.
\item may be ruled out by principled considerations as well. Perhaps,
  world-abstractors are only allowed at sentential boundaries.
\end{enumerate}
%
See \citet{larson:grammar-intensionality} for some discussion of recalcitrant
cases, one of which is the object position of so-called intensional transitive
verbs, a topic for another occasion.

\section{Further reading}

{\setlength{\parindent}{0pt}\setlength{\parskip}{6pt}

There is an interesting literature spawned by \cite{percus-2000-constraints}:
  
\begin{bibentrylist}
\item \fullcite{schwager-2009-qualities}.
\item \fullcite{romoli-sudo-2009-dere}.
\item \fullcite{keshet-2010-economy}.
\item \fullcite{keshet-2011-split}.
\item \fullcite{schwarz-2012-situation-pronouns}.
\item \fullcite{keshet-schwarz-2014-de-re-de-dicto}.
\end{bibentrylist}

Fodor also discussed a fourth reading, specific opaque, which is hard to fit
into our framework. Whether it really exists is a question discussed in some
recent work:

\begin{bibentrylist}
\item \fullcite{szabo-2010-specific-opaque}.
\item \fullcite{francez-2017-summative}.
\end{bibentrylist}

}


%!TEX root = IntensionalSemantics.tex
\chapter{Ordering}\label{cha:ordering} % (fold)

\chapterprecishere{We have stressed throughout the previous two chapters that there are numerous parallels between quantification over ordinary individuals via determiner quantifiers and quantification over possible worlds via modal operators (including conditionals). Now, we turn to a phenomenon that (at least at first glance) appears to show that there are non-parallels as well: a sensitivity to an \term{ordering} of the elements in the domain of quantification. We first look at this in the context of simple modal sentences and then we look at conditionals.}

\minitoc

\section{The Driveway}

Consider a typical use of a sentence like \Next.

\ex. \label{drivewayfine}John must pay a fine.

This is naturally understood in such a way that its truth depends both on facts about the law and facts about what John has done. For instance, it will be judged true if (i) the law states that driveway obstructors are fined, and (ii) John has obstructed a driveway. It may be false either because the law is different or because John's behavior was different.

What accessibility relation provides the implicit restriction of the quantifier \expression{must} on this reading of \Last? A naïve attempt might go like this:

\ex. $\lambda w.\ \lambda w'.$ [what happened in $w'$ up to now is the same as what happened in $w$ and $w'$ conforms to what the law in $w$ demands].

The problem with \Last is that, unless there were no infractions of the law at all in $w$ up to now, no world $w'$ will be accessible from $w$. Therefore, \LLast is predicted to follow logically from the premise that John broke some law. This does not represent our intuition about its truth conditions.

A better definition of the appropriate accessibility relation has to be more complicated:

\ex. $\lambda w.\ \lambda w'.$ [what happened in $w'$ up to now is the same as what happened in $w$ and $w'$ conforms at least as well to what the law in $w$ demands as does any other world in which what happened up to now is the same as in $w$].

\Last makes explicit that there is an important difference between the ways in which facts about John's behavior on the one hand, and facts about the law on the other, enter into the truth conditions of sentences like \ref{drivewayfine}. Worlds in which John didn't do what he did are simply excluded from the domain of \expression{must} here. Worlds in which the law isn't obeyed are not absolutely excluded. Rather, we restrict the domain to those worlds in which the law is obeyed as well as it can be, considering what has happened. We exclude only those worlds in which there are infractions above and beyond those that are shared by all the worlds in which John has done what he has done. The analysis of \ref{drivewayfine} thus crucially involves the notion of an ordering of worlds: here they are ordered according to how well they conform to what the law in $w$ demands.

\section{Kratzer's Solution: Doubly Relative Modality}

Kratzer proposes that modal operators are sensitive to \emph{two} context-dependent parameters: a set of accessible worlds (provided by an accessibility function computed from a conversational background, the \term{modal base}), and a partial ordering of the accessible worlds (computed from another conversational background, called the \term{ordering source}).

Let's see how the analysis applies to the previous example.

\begin{itemize}
	\item The modal base will be a function that assigns to any evaluation world a set of propositions describing the relevant circumstances, for example, what John did. Since in our stipulated evaluation world John obstructed a driveway, the modal base will assign the proposition that John obstructed a driveway to this world. The set of worlds accessible from the evaluation world will thus only contain worlds where John obstructed a driveway. 
	\item The ordering source will be a function that assigns to any evaluation world a set of propositions $\mathcal{P}$ whose truth is demanded by the law. Imagine that for our evaluation world this set of propositions contains (among others) the following two propositions: (i) nobody obstructs any driveways, (ii) anybody who obstructs a driveway pays a fine. 
	\item The idea is now that such a set $\mathcal{P}$ of propositions can be used to order the worlds in the modal base. For any pair of worlds $w_1$ and $w_2$, we say that $w_1$ comes closer than $w_2$ to the ideal set up by $\mathcal{P}$ (in symbols: $w_1 <_{\mathcal{P}} w_2$), iff the set of propositions from $\mathcal{P}$ that are true in $w_2$ is a proper subset of the set of propositions from $\mathcal{P}$ that are true in $w_1$. 
	\item For our simple example then, any world in modal base where John pays a fine will count as better than an otherwise similar world where he doesn't. 
	\item Modals then make quantificational claims about the best worlds in the modal base (those for which there isn't a world that is better than them). 
	\item In our case, \ref{drivewayfine} claims that in the best worlds (among those where John obstructed a driveway), he pays a fine. 
\end{itemize}
%
More technically:

\ex. Given a set of worlds $X$ and a set of propositions $\mathcal{P}$, define the \term{strict partial order} $<_P$ as follows:\\
$\forall w_1,w_2 \in X\co w_1 <_{\mathcal{P}} w_2 \mbox{ iff } \{p \in \mathcal{P}\co\ p(w_2)=1\} \subset \{p \in \mathcal{P}:\ p(w_1)=1\}$.

\ex. For a given strict partial order $<_{\mathcal{P}}$ on worlds, define the selection function max$_{\mathcal{P}}$ that selects the set of $<_{\mathcal{P}}$-best worlds from any set X of worlds:\\
$\forall X \subseteq W\co \mbox{max}_{\mathcal{P}}(X) = \left\{ w \in X\co \neg \exists w' \in X\co w' <_{\mathcal{P}} w \right\}$.

\ex. $\sv{\mbox{must}}^{w,g} = \lambda f_{\angles{s,st}}.\ \lambda g_{\angles{s,st}}.\ \lambda q_{\angles{s,t}}.\\
\null\hfill \forall w' \in \mbox{max}_{g(w)} (\cap f(w))\co q(w') = 1$.

{\scshape Technical Note}: This only works if we can in general assume that the $<_P$ relation has minimal elements, that there always are accessible worlds that come closest to the $P$-ideal, worlds that are better than any world they can be compared with via $<_P$. It is possible, with some imagination, to cook up scenarios where this assumption fails. This problem has been discussed primarily in the area of the semantics of conditionals. There, Lewis presents relevant scenarios and argues that one shouldn't make this assumption, which he calls the Limit Assumption. Stalnaker, on the one other hand, defends the assumption against Lewis' arguments by saying that in actual practice, in actual natural language semantics and in actual modal/conditional reasoning, the assumption is eminently reasonable. Kratzer is persuaded by Lewis' evidence and does not make the Limit Assumption; hence her semantics for modals is more convoluted than what we have in \LLast and \Last. I will side with Stalnaker, not the least because it makes life easier. For further discussion, see \citet[19--21]{lewis:1973:counterfactuals} and \citet[Chapter 7, esp. pp. 140--142]{stalnaker:1984:inquiry}; \citet{pollock:1976:subjunctive}, \citet{herzberger:1979:consistency}, and \citet{warmbrod:1982:limit} argue for the Limit Assumption as well.

% \section{How to Get to Harvard Square}
% \label{sec:how-get-harvard}
\begin{exercise}
	
	In her handbook article \cite{kratzer:1991:modality}, Kratzer presents a number of examples of modal statements and sketches an analyses in terms of doubly relative modality. You should study her examples carefully. \eex
\end{exercise}

%
% Here is another example:
% \begin{itemize}
% \item The modal base is circumstantial. It assigns to the evaluation
%   world a set of propositions describing the relevant
%   circumstances. Imagine that for our world, the following facts are
%   relevant: (i) you are in Kendall Square, (ii) you can get to Harvard
%   Square on the Red Line, by taxi, or on foot, (iii) the Red Line
%   costs \$1.25, takes 10 minutes, and is safe, (iv) the taxi costs
%   \$10, is fast, and is safe, (v) walking is free, slow, and unsafe.
% \item The ordering source describes your goals (we could call this a
%   \term{teleological} ordering source). Your relevant goals in our
%   world are: (i) you get to Harvard Square, (ii) you pay less than
%   \$2, (iii) you are safe.
% \item What are the best worlds in the modal base according to the
%   given ordering source? The worlds where you take the Red Line.
% \item That's why it seems true to say:
% \ex. Given your goals, you ought to take the Red Line.
% \end{itemize}
\section{The Paradox of the Good Samaritan} \label{sec:parad-good-samar}

\citet{prior:1958:escapism} introduced the following ``Paradox of the Good Samaritan''. Imagine that someone has been robbed and John is walking by. It is easy to conceive of a code of ethics that would make the following sentence true:

\ex. John ought to help the person who was robbed.

In our previous one-factor semantics for modals, we would have said that \Last says that in all of the deontically accessible worlds (those compatible with the code of ethics) John helps the person who was robbed. Prior's point was that under such a semantics, something rather unfortunate holds. Notice that in all of the worlds where John helps the person who was robbed, someone was robbed in the first place. Therefore, it will be true that in all of the deontically accessible worlds, someone was robbed. Thus, \Last will entail:

\ex. It ought to be the case that someone was robbed.

It clearly would be good not make such a prediction.

The doubly-relative analysis of modality can successfully avoid this unfortunate prediction. We conceive of \LLast as being uttered with respect to a circumstantial modal base that includes the fact that someone was robbed. Among those already somewhat ethically deficient worlds, the relatively best ones are all worlds where John helps the victim. 

Note that we still have the problematic fact that among the worlds in the modal base, all are worlds where someone was robbed, and we would thus appear to still make the unfortunate prediction that \Last should be true. But this can now be fixed. For example, we could say that \emph{ought $p$} is semantically defective if $p$ is true throughout the worlds in the modal base. This could be a presupposition or some other ingredient of meaning. So, with respect to a modal base which pre-determines that someone was robbed, one couldn't felicitously say \Last. 

Consequently, saying \Last would only be felicitous if a different modal base is intended, one that contains both $p$ and non-$p$ worlds. And given a choice between worlds where someone was robbed and worlds where nobody was robbed, most deontic ordering sources would probably choose the no-robbery worlds, which would make \Last false, as desired.


\subsubsection{Kratzer's version of the Samaritan Paradox}

\citet{kratzer:1991:modality} argues that the restrictor approach to deontic conditionals is the crucial ingredient in the solution to a conditional version of the Samaritan Paradox:

\ex. If a murder occurs, the jurors must convene.

Kratzer points out that if one tried to analyze \Last as a material implication embedded under deontic necessity, then one quickly runs into a problem. Surely, one wants the following to be a true statement about the law:

\ex. There must be no murder.                           

But this means that in the deontically accessible worlds, all of them have no murders occurring. Now, this means that in all of the deontically accessible worlds, any material implication of the form ``if a murder occurs, $q$'' will be true no matter what the consequent is since the antecedent will be false. Since that is an absurd prediction, \LLast cannot be analyzed as material implication under deontic necessity. The combination of the restrictor approach to \emph{if}-clauses and the doubly-relative theory of modals can rescue us from this problem. \Last is analyzed as the deontic necessity modal being restricted by the \emph{if}-clause. The set of accessible worlds is narrowed down by the \emph{if}-clause to only include worlds in which a murder occurs. The deontic ordering then identifies the best among those worlds and those are plausibly all worlds where the jurors convene.

\section{Non-Monotonicity of Conditionals}

The last case discussed takes us straight to the crucial role of the ordering of worlds in the semantics of conditionals, as we would of course expect under the analysis of \expression{if}-clauses as restrictors of modal operators. In this arena, the discussion usually revolves around the failure of certain inference patterns, which one would expect a universal quantifier to validate. Here are the most important ones:

\ex. \extitle{Left Downward Monotonicity (``Downward Entailingness'')}\\[3pt]
Every A is a B. $\rightarrow$ Every A \& C is a B.

\ex. \extitle{Transitivity}\\[3pt]
Every A is a B. Every B is a C. $\rightarrow$ Every A is a C.

\ex. \extitle{Contraposition}\\[3pt]
Every A is a B. $\rightarrow$ Every non-B is a non-A.

Conditionals were once thought to obey these patterns as well, known in conditional logic as \term{Strengthening the Antecedent}, \term{Hypothetical Syllogism}, and \term{Contraposition}. But then spectacular counterexamples became known through the work of Stalnaker and Lewis.

\ex. \extitle{Failure of Strengthening the Antecedent} \a. If I strike this match, it will light.\\
If I dip this match into water and strike it, it will light. \b. \label{SAb}If John stole the earrings, he must go to jail.\\
If John stole the earrings and then shot himself, he must go to jail. \c. If kangaroos had no tails, they would topple over. If kangaroos had no tails but used crutches, they would topple over.

\ex. \extitle{Failure of the Hypothetical Syllogism (Transitivity)} \a. If Brown wins the election, Smith will retire to private life.\\
If Smith dies before the election, Brown will win the election.\\
If Smith dies before the election,Smith will retire to private life. \b. If Hoover had been a Communist, he would have been a traitor.\\
If Hoover had been born in Russia, he would have been a Communist.\\
If Hoover had been born in Russia, he would have been a traitor.

\ex. \extitle{Failure of Contraposition} \a. If it rained, it didn't rain hard.\\
If it rained hard, it didn't rain. \b. (Even) if Goethe hadn't died in 1832, he would still be dead now.\marginnote{The Goethe example is due to Kratzer.}\\
If Goethe were alive now, he would have died in 1832. 

Note that these cases involve examples of both ``indicative'' (epistemic) conditionals and counterfactual conditionals. It is sometimes thought that indicative conditionals are immune from these kinds of counterexamples, but it is clear that they are not. Also note that in \ref{SAb} we have a case of Failure of Strengthening the Antecedent with a deontic conditional. Deontic counterexamples to the other patterns seem harder to find.

The failure of these inference patterns again indicates that the semantics of modal operators (restricted by \expression{if}-clauses) is more complicated than the simple universal quantification we had previously been assuming. The basic idea of most approaches to this problem is this: the semantics of conditionals is more complicated than simple universal quantification. The conditional does not make a claim about simply every antecedent world, nor even about every contextually relevant antecedent world. Instead, in each of the conditional statements, only a particular subset of the antecedent worlds is quantified over. Informally, we can call those the ``most highly ranked antecedent worlds''. Consider:

\ex. If I had struck this match, it would have lit.\\
If I had dipped this match into water and struck it, it would have lit.

According to the Stalnaker-Lewis account, this inference is semantically invalid. The premise merely claims that the most highly ranked worlds in which I strike this match are such that it lights. No claim is made about the most highly ranked worlds in which I first dip this match into water and then strike it. Strengthening the Antecedent will only be safe if it is additionally known that the strengthened antecedent is instantiated among the worlds that verify the original antecedent.

The other fallacies receive similar treatments. Transitivity (Hypothetical Syllogism) fails for the new non-monotonic quantifier because even if all the most highly rated $p$-worlds are $q$-worlds and all the most highly rated $q$-worlds are $r$-worlds, we are not necessarily speaking about the same $q$-worlds (the $q$-worlds that $p$ takes us to may be rather remote ones). So in the Hoover-example, we get the following picture: The most highly ranked $p$-worlds in which Hoover was born in Russia (but where he retains his level of civic involvement), are all $q$-worlds in which he becomes a Communist. On the other hand, the most highly ranked $q$-worlds in which he is a Communist (but retaining his having been born in the United States and being a high level administrator) are all $r$-worlds in which he is a traitor. However, the most highly ranked $p$-worlds do not get us to the most highly ranked $q$-worlds, so the Transitive inference does not go through.

Contraposition fails because the fact that the most highly rated $p$-worlds are $q$-worlds does not preclude a situation where the most highly rated non $q$-worlds are also $p$-worlds. The most highly rated $p$-worlds in which Goethe didn't die in 1832 are all $q$-worlds where he dies nevertheless (well) before the present. But of course, the most highly rated (in fact, all) non-$q$-worlds (where he is alive today) are also $p$-worlds where he didn't die in 1832.

In the conditionals literature, the ordering of worlds is usually given directly as an evaluation parameter. The typical gloss is that the ordering ranks possible worlds based on how \emph{similar} they are to the evaluation world. Kratzer developed an alternative where the ordering is computed from a set of propositions true in the evaluation world. \citet{lewis:1981:ordering} showed that ordering semantics and premise semantics are largely notational variants.

\section*{Supplementary Readings} \label{sec:suppl-read-ordering}

\phantomsection \addcontentsline{toc}{section}{Supplemental Readings}

{\setlength{\parindent}{0pt}\setlength{\parskip}{6pt}

The central readings for this chapter are two papers by Kratzer:

\begin{bibentrylist}
	\item \bibentry{kratzer:1991:modality}. 
	\item \bibentry{kratzer:1981:notional}. 
\end{bibentrylist}

Some work that discusses and uses Kratzer's two factor semantics for modals:

\begin{bibentrylist}
	\item \bibentry{frank:1996:dissertation}. 
	\item \bibentry{fintel-iatridou:2005:harlem}. 
	\item \bibentry{fintel-iatridou:2008:ought}.
\end{bibentrylist}

Some work that discusses whether non-monotonicity could be or might have to be relegated to a dynamic pragmatic component of meaning:

\begin{bibentrylist}
	\item \bibentry{fintel:2001:counterfactuals}. 
	\item \bibentry{fintel:1999:npi}.
	\item \bibentry{gillies:2007:scorekeeping}.
\end{bibentrylist}

Schlenker explored whether the apparent non-monotonicity in conditional is paralleled in quantification over individuals:

\begin{bibentrylist}
	\item \bibentry{schlenker:2004:conditionals}.
\end{bibentrylist}

% ADDITIONS
% 
% Arregui on should
% Kolodny & MacFarlane 
% 
% Rullmann, Matthewson, Davis 2008

% Literature on imperatives

% Literature on dynamic effects with modals, conditionals

% chapter ordering (end)

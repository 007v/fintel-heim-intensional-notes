\frontmatter

\title{\caps{Intensional Semantics}}
\author{\caps{Kai von Fintel}\and{\caps{Irene Heim}}} 
\date{\caps{MIT Spring 2015 Edition}}

\pagestyle{empty}

\maketitle

\clearpage

\section*{About these lecture notes}

\plainbreak{1} 

These lecture notes have been evolving for years now, starting with
some old notes from the early 1990s by Angelika Kratzer, Irene Heim,
and myself, which have since been modified and expanded many times by
Irene or myself. Because this version of the notes has not been seen
by my co-author, I alone am responsible for any defects.

\plainbreak{1} 

We encourage the use of these notes in courses at other institutions.
Of course, you need to give full credit to the authors and you may not
use the notes for any commercial purposes. If you use the notes, we
would like to be notified and we would very much appreciate any
comments, criticism, and advice on these materials.

\plainbreak{1}

Link to the latest full version (currently the 2011 edition):\\[6pt]
                                             \null\hfill\url{http://kvf.me/intensional}
\medskip
                                            
\noindent GitHub repository with the current development version:\\[6pt]
            \null\hfill\url{https://github.com/fintelkai/fintel-heim-intensional-notes}

\medskip

\noindent Homepage of the class these notes are designed for:\\[6pt]
                        \null\hfill\url{http://stellar.mit.edu/S/course/24/sp15/24.973}

\vfill

\noindent Kai von Fintel\\
Department of Linguistics \amp\ Philosophy\\
Room 32\textperiodcentered{}\textsc{d}808\\
Massachusetts Institute of Technology\\
77 Massachusetts Avenue\\
Cambridge, \textsc{ma} 02139-4307\\
\textsc{United States of America} 

\plainbreak{1}

\href{mailto:fintel@mit.edu}{fintel@mit.edu}\\
\url{http://kaivonfintel.org} 

\cleardoublepage

\null
\vfill \ba 

\section*{Some advice}
\begin{enumerate}
\item These notes presuppose familiarity with the material, concepts,
  and notation of the Heim \amp\ Kratzer textbook.
\item There are numerous exercises throughout the notes. It is highly
  recommended to do all of them and it is certainly necessary to do so
  if you at all anticipate doing semantics-related work in the future.
\item The notes are designed to go along with explanatory lectures.
  You should ask questions and make comments as you work through the
  notes.
\item Students with semantic ambitions should also at an early point
  start reading supplementary material (as for example listed at the
  end of each chapter of these notes).
\item Lastly, prospective semanticists may start thinking about how
  \emph{they} would teach this material.
\end{enumerate}

\ab 
\vfill\null

\newpage\hbox{}
\vfill{\scshape\caps{--- This page intentionally left blank ---}}
\vfill\hbox{}\thispagestyle{cleared}
\excnt=1
\chapter{Specificity and Transparency}

\chapterprecishere{We discuss two important aspects of the interpretation of DPs
  in intensional contexts: the scope of their quantificational force, if any,
  and the world with respect to which their predicate is evaluated.}

\minitoc

\section{Predictions of our framework}
\label{sec:predictions}

When a DP occurs in the scope of an intensional operator, our framework makes
clear predictions. Consider, for example:

\ex
Chris wanted Dana to buy a book about soccer.
\xe
%
Imagine that we give the following meaning to \emph{want}:

\ex
$\svt{want}^{w,g} = \lambda p_{\type{s,t}}.\lambda x_e.\ \forall w'$ such that
$x$'s wants in $w$ are satisfied in $w'\co p(w')=1$. 
\xe
%
In other words, \emph{$x$ wants $p$} is true iff $p$ is true in all worlds where
$x$'s wants are satisfied. Further, assume that the DP \emph{a book about
  soccer} is interpreted within the embedded clause. Then, we claim, (\blastx)
will be true iff in all of the worlds that satisfy all of Chris' wants, there is
a book about soccer that Dana buys. (You prove this claim in the following
exercise.)

\begin{exercise}
  Draw the obvious, if simplified, LF for (\blastx) and calculate its
  truth-conditions. \eex
\end{exercise}
%
Now, consider what happens if the object of the lower verb QRs and adjoins to
the matrix clause:

\ex\null
[a book about soccer] (1 [Chris wanted Dana to buy $t_1$])
\xe
%
When you calculate the truth-conditions of (\lastx) [please do so], you will get
a result that is very different from the previous exercise. Now what is claimed
is that there is a book about soccer, call it $x$, such that in all of the
worlds satisfying all of Chris' wants Dana buys $x$.

There are two important differences between the truth-conditions that our framework
assigns to these two LFs.

\emph{Quantifier scope}: Since \emph{want} is a universal quantifier over
worlds and \emph{a book about soccer} is an existential quantifier over
individuals, there's a question about the relative scope of the two quantifiers.
In the first truth-conditions we sketched, the existential quantifier scopes
under the universal quantifier: for every world there is an individual such that
bla-bla. In the second LF, the existential quantifier scopes over the universal
one: there is an individual such that in every world yadda-yadda. The most
common terminology for this difference involves the pair
\emph{specific/non-specific}: the sentence (or the object DP) is interpreted
specifically if the DP takes scope over the intensional operator, and it is
interpreted non-specifically if the DP takes scope under the intensional operator.

\emph{Predicate evaluation}: When the existential quantifier scopes over the
intensional operator, this also has the effect that the predicate contained in
it, \emph{book about soccer}, is evaluated in the matrix evaluation world. And
when the quantifier takes lower scope, its predicate is evaluated in the worlds
that the intensional operator shifts to. One evocative terminology for whether
the predicate is evaluated with respect to the matrix evaluation world or the
worlds shifted to by the intensional operator is \emph{transparent/opaque}. A
predicate evaluated relative to the matrix world is called \emph{transparent}. A
predicate evaluated in the worlds shifted to by an intensional operator is
receiving an \emph{opaque} interpretation.

There's another terminological pair that is very common: \emph{de re}/\emph{de
  dicto}. One way to conceive of that distinction in our framework is that it
stands for a particular combination of the two distinctions we just introduced:
\emph{de re} means specific and transparent, and \emph{de dicto} means
non-specific and opaque.

Caution about terminology: terminological confusion and exuberance is rampant in
this area (and many others). In a way, terminology is just a short-hand way to
pick out salient properties of LFs (or their denotation). It's the latter that
truly matters. One particular problematic aspect of the terminology is its
binary nature, while the relevant distinctions are actually more complex,
especially as soon as we are dealing with nested intensionality.

\begin{exercise}
  Consider the sentence \emph{Chris must want Dana to buy a book about soccer}.
  One can imagine using this to describe a scenario where we are seeing Dana
  enter a bookstore known to cater to soccer aficionados. For some reason we
  won't go into, we come to the conclusion that there is a specific book about
  soccer that Chris must have asked Dana to buy. But at the same time, we have
  no idea what that book might be, so there's not a specific book about which we
  made our deduction. This suggest that we may want to give the object DP
  intermediate scope. So, draw an LF that corresponds to this idea and calculate
  its truth-conditions. \eex
\end{exercise}

Our recommendation is to use the terms \emph{specific/non-specific},
\emph{transparent/opaque}, \emph{de re/de dicto} only with extreme caution. They
are sometimes useful shorthands, but unless it is crystal-clear what properties
of LFs/denotations you are using them to pick out, they are more likely to be a
source of obfuscation and confusion.

Let's look at some more examples of the ambiguity predicted by our framework as
soon as we allow for the possibility of an embedded DP to take scope either
under or over a relevant intensional operator. A classic kind of example is
(\nextx), which contains the DP \emph{a plumber} inside the infinitive
complement of \emph{want}.

\ex John wants to marry a plumber. \xe
%
According to the non-specific reading, every possible world in which John gets
what he wants is a world in which there is a plumber whom he marries. According
to the specific reading, there is a plumber in the actual world whom John
marries in every world in which he gets what he wants. We can imagine situations
in which one of the readings is true and the other one false.

For example, suppose John thinks that plumbers make ideal spouses, because they
can fix things around the house. He has never met one so far, but he definitely
wants to marry one. In this scenario, the non-specific reading is true, but the
specific reading is false. What all of John's desire-worlds have in common is
that they have a plumber getting married to John in them. But it's not the same
plumber in all those worlds. In fact, there is no particular individual (actual
plumber or other) whom he marries in every one of those worlds.

For a different scenario, suppose that John has fallen in love with Robin and
wants to marry Robin. Robin happens to be a plumber, but John doesn't know this;
in fact, he wouldn't like it and might even call off the engagement if he found
out. Here the specific reading is true, because there is an actual plumber, viz.
Robin, who gets married to John in every world in which he gets what he wants.
The non-specific reading is false, however, because the worlds which conform to
John's wishes actually do not have him marrying a plumber in them. In his
favorite worlds, he marries Robin, who is not a plumber in those worlds.

When confronted with this second scenario, you might, with equal justification,
say `John wants to marry a plumber', or `John \emph{doesn't} want to marry a
plumber'. Each can be taken in a way that makes it a true description of the
facts \dash although, of course, you cannot assert both in the same
breath.\marginnote{Actually, \emph{why} wouldn't one be able to assert both
  sentences in the same breath, if both have a true reading?} This intuition
fits well with the idea that we are dealing with a genuine ambiguity.

Let's look at another example:

\ex John believes that your abstract will be accepted. \xe
%
Here the relevant DP in the complement clause of the verb \emph{believe} is
\emph{your abstract}. Again, we detect an ambiguity, which is brought to light
by constructing different scenarios.

\begin{enumerate}[(i)]
	
\item John's belief may be about an abstract that he reviewed, but since the
  abstract is anonymous, he doesn't know who wrote it. He told me that there was
  a wonderful abstract about subjacency in Hindi that is sure to be accepted. I
  know that it was your abstract and inform you of John's opinion by saying
  (\lastx). This is the specific reading. In the same situation, the non-specific
  reading is false: Among John's belief worlds, there are many worlds in which
  \emph{your abstract will be accepted} is not true or even false. For all he
  knows, you might have written, for instance, that terrible abstract about
  Antecedent-Contained Deletion, which he also reviewed and is positive will be
  rejected.
	
\item For the other scenario, imagine that you are a famous linguist, and John
  doesn't have a very high opinion about the fairness of the abstract selection
  process. He thinks that famous people never get rejected, however the
  anonymous reviewers judge their submissions. He believes (correctly or
  incorrectly \dash this doesn't matter here) that you submitted a (unique)
  abstract. He has no specific information or opinion about the abstract's
  content and quality, but given his general beliefs and his knowledge that you
  are famous, he nevertheless believes that your abstract will be accepted. This
  is the non-specific reading. Here it is true in all of John's belief worlds
  that you submitted a (unique) abstract and it will be accepted. The specific
  reading of (\lastx), though, may well be false in this scenario. Suppose \dash to
  flesh it out further \dash the abstract you actually submitted is that terrible
  one about ACD. That one surely doesn't get accepted in every one of John's
  belief worlds. There may be some where it gets in (unless John is certain it
  can't be by anyone famous, he has to allow at least the possibility that it
  will get in despite its low quality). But there are definitely also
  belief-worlds of his in which it doesn't get accepted.
	
	We have taken care here to construct scenarios that make one of the readings
  true and the other false. This establishes the existence of two distinct
  readings. We should note, however, that there are also many possible and
  natural scenarios that simultaneously support the truth of \emph{both}
  readings. Consider, for instance, the following third scenario for sentence
  (\lastx).
	
\item John is your adviser and is fully convinced that your abstract will be
  accepted, since he knows it and in fact helped you when you were writing it.
  This is the sort of situation in which both the non-specific and the specific
  reading are true. It is true, on the one hand, that the sentence \emph{your
    abstract will be accepted} is true in every one of John's belief worlds
  (non-specific reading). And on the other hand, if we ask whether the abstract
  which you actually wrote will get accepted in each of John's belief worlds,
  that is likewise true (specific reading).
	
	In fact, this kind of ``doubly verifying'' scenario is very common when we
  look at actual uses of attitude sentences in ordinary conversation. There may
  even be many cases where communication proceeds smoothly without either the
  speaker or the hearer making up their minds as to which of the two readings
  they intend or understand. It doesn't matter, since the possible circumstances
  in which their truth-values would differ are unlikely and ignorable anyway.
  Still, we \emph{can} conjure up scenarios in which the two readings come
  apart, and our intuitions about those scenarios do support the existence of a
  semantic ambiguity.
  
\end{enumerate}

\begin{exercise}

For the two examples just discussed, we can explain their non-specific (and
opaque) interpretation via LFs where the relevant DP remains inside the scope of
the intensional operator at LF:

\ex\label{first} John wants [ [ a plumber]$_1$ [ \textsc{pro}$_2$ to marry
t$_1$]] \xe

\ex John believes [ the abstract-by-you will-be-accepted] \xe

To obtain the specific (and transparent) readings, we apparently have to QR the
DP to a position above the intensional predicate, minimally the VP headed by
\emph{want} or \emph{believe}.

\ex\null [ a plumber]$_1$ [ John wants [ \textsc{pro}$_2$ to marry t$_1$]] \xe

\ex\null [ the abstract-by-you]$_1$ [ John believes t$_1$ will-be-accepted] \xe
%	
Calculate the interpretations of the four structures in \refx{first}\dash(\lastx),
and determine their predicted truth-values in each of the (types of) possible
worlds that we described above in our introduction to the ambiguity.
	
Some assumptions to make the job easier: (i) Assume that \refx{first} and (\blastx)
are evaluated with respect to a variable assignment that assigns John to the
number 2. This assumption takes the place of a worked out theory of how
controlled PRO is interpreted. (ii) Assume that \emph{abstract-by-you} is an
unanalyzed one-place predicate. This takes the place of a worked out theory of
how genitives with a non-possessive meaning are to be analyzed. \eex
\end{exercise}

\section{Raised subjects}
\label{sec:raised}

In the examples of ambiguities that we have looked at so far, the surface
position of the DP in question was inside the modal predicate's clausal or
VP-complement. We saw that if it stays there at LF, a non-specific opaque
reading results, and if it covertly moves up above the modal operator, we get a
specific transparent reading. In the present section, we will look at cases in
which a DP that is superficially \emph{higher} than a modal operator can still
be read non-specifically. In these cases, it is the specific reading which we
obtain if the LF looks essentially like the surface structure, and it is the
non-specific reading for which we apparently have to posit a non-trivial covert
derivation.

\subsection{Non-specific readings for raised subjects}

Suppose I come to my office one morning and find the papers and books on my desk
in different locations than I remember leaving them the night before. I say:

\ex \label{some} Somebody must have been here (since last night).\xe
%
On the assumptions we have been making, \emph{somebody} is base-generated as the
subject of the VP \emph{be here} and then moved to its surface position above
the modal. So (\lastx) has the following S-structure, which is also an
interpretable LF.

\ex \label{dere} somebody [ 2 [ [ must $R$] [ t$_2$ have-been-here]]] \xe
%
What does (\lastx) mean? The appropriate reading for \emph{must} here is
epistemic, so suppose the variable $R$ is mapped to the relation $\bigl[\lambda
w.\lambda w'.\ w'$ is compatible with what I believe in $w\bigr]$. Let $w_{0}$
be the utterance world. Then the truth-condtion calculated by our rules is as
follows.

\ex $\exists x [x$ is a person in $w_{0}$ \& \\
$\forall w'[w'$ is compatible with what I believe in $w_{0}$ $\rightarrow\ x$ was here in $w'$]] \xe
%
But this is not the intended meaning. For (\lastx) to be true, there has to be a
person who in every world compatible with what I believe was in my office. In
other words, all my belief-worlds have to have one and the same person coming to
my office. But this is not what you intuitively understood me to be saying about
my belief-state when I said \refx{some}. The context we described suggests that I
do not know (nor have any opinion about) which person it was that was in my
office. For all I know, it might have been John, or it might have been Mary, or
it have been this stranger here, or that stranger there. In each of my
belief-worlds, somebody or other was in my office, but no one person was there
in all of them. I do not believe of anyone in particular that he or she was
there, and you did not understand me to be saying so when I uttered \refx{some}.
What you did understand me to be claiming, apparently, was not (\lastx) but
(\nextx).

\ex \label{dedic} $\forall w' [ w'$ is compatible with what I believe in $w_{0}$\\
\null\hfill$\rightarrow\ \exists x\ [x$ is a person in $w'$ \& $x$ was here in $w']]$ \xe
%
In other words \dash to use the terminology we introduced in the last section
\dash the DP \emph{somebody} in \refx{some} appears to have a non-specific
reading.

How can sentence \refx{some} have the meaning in (\lastx)? The LF in
\refx{dere}, as we saw, means something else; it expresses a specific reading,
which typically is false when \refx{some} is uttered sincerely. So there must be
another LF. What does it look like and how is it derived? One way to capture the
intended reading, it seems, would be to generate an LF that's essentially the
same as the underlying structure we posited for \refx{some}, i.e., the structure
\emph{before} the subject has raised:

\ex\ [$_{\text{IP}}$ e [$_{\text{I}'}$ [ must $R$] [ somebody have-been-here]]] \xe
%
(\lastx) means precisely (\blastx) (assuming that the unfilled Spec-of-IP
position is semantically vacuous), as you can verify by calculating its
interpretation by our rules. So is (\lastx) (one of) the LF(s) for \refx{some},
and what assumption about syntax allow it to be generated? Or are there other
\dash perhaps less obvious, but easier to generate \dash candidates for the
non-specific LF-structure of \refx{some}?

Before we get into these question, let's look at a few more examples. Each of
the following sentences, we claim, has a non-specific reading for the subject,
as given in the accompanying formula. The modal operators in the examples are of
a variety of syntactic types, including modal auxiliaries, main verbs,
adjectives, and adverbs.

\ex\label{everymay} Everyone in the class may have received an A.\\
$\exists w'[w'$ conforms to what I believe in $w$ \&\\
\null\hfill$\forall x[x$ is in this class in $w'\ \rightarrow\ x$ received an A in $w'$]].\xe

\ex At least two semanticists have to be invited.\\
$\forall w'[w'$ conforms to what is desirable in $w$\\
\null\hfill$ \rightarrow\ \exists_2 x$\ [$x$ is a semanticist in $w'$ \& $x$ is invited in $w'$]].\xe

\ex\label{ny} Somebody from New York is expected to win the lottery.\\
$\forall w'[w'$ conforms to what is expected in $w$\\
\null\hfill$ \rightarrow\ \exists x[x$ is a person from NY in $w'$ \& $x$ wins the lottery in $w'$]]\xe

\ex\label{nyli} Somebody from New York is likely to win the lottery.\\
$\forall w'[w'$ is as likely as any other world, given what I know in $w$\\
\null\hfill$ \rightarrow\ \exists x[x$ is a person from NY in $w'$ \& $x$ wins the lottery in $w'$]]\footnote{Hopefully the exact analysis of the modal operators \emph{likely} and \emph{probably} is not too crucial for the present discussion, but you may still be wondering about it. As you see in our formula, we are thinking of \emph{likely} (\emph{probably}) as a kind of epistemic necessity operator, i.e., a universal quantifier over a set of worlds that is somehow determined by the speaker's knowledge. (We are focussing on the ``subjective probability'' sense of these words. Perhaps there is also an ``objective probability'' reading that is circumstantial rather than epistemic.) What is the difference then between \emph{likely} and e.g. epistemic \emph{must} (or \emph{necessary} or \emph{I believe that})? Intuitively, `it is likely that p' makes a weaker claim than `it must be the case that p'. If both are universal quantifiers, then, it appears that \emph{likely} is quantifying over a smaller set than \emph{must}, i.e., over only a proper subset of the worlds that are compatible with what I believe. The difference concerns those worlds that I cannot strictly rule out but regard as remote possibilities. These worlds are included in the domain for \emph{must}, but not in the one for \emph{likely}. For example, if there was a race between John and Mary, and I am willing to bet that Mary won but am not completely sure she did, then those worlds where John won are remote possibilities for me. They are included in the domain of \emph{must}, and so I will not say that Mary \emph{must} have won, but they are not in the domain quantified over by \emph{likely}, so I do say that Mary is \emph{likely} to have won.\\
  \indent This is only a very crude approximation, of course. For one thing, probability is a gradable notion. Some things are more probable than others, and where we draw the line between what's probable and what isn't is a vague or context-dependent matter. Even \emph{must}, \emph{necessary} etc. arguably don't really express complete certainty (because in practice there is hardly anything we are completely certain of), but rather just a very high degree of probability. For more discussion of \emph{likely}, \emph{necessary}, and other graded modal concepts in a possible worlds semantics, see e.g. Kratzer 1981, 1991.\\
  \indent A different approach may be that \emph{likely} quantifies over the
  same set of worlds as \emph{must}, but with a weaker, less than universal,
  quantificational force. I.e., `it is likely that p' means something like p is
  true in \emph{most} of the worlds conforming to what I know. A \emph{prima
    facie} problem with this idea is that presumably every proposition is true
  in infinitely many possible worlds, so how can we make sense of cardinal
  notions like `more' and `most' here? But perhaps this can be worked out
  somehow. }\xe

\ex One of these two people is probably infected.\\
$\forall w'[w'$ is as likely as any other world, given what I know in $w$\\
\null\hfill$ \rightarrow\ \exists x[x$ is one of these two people \& $x$ is in infected in $w'$]]\xe

To bring out the intended non-specific reading of the last example (to pick just
one) imagine this scenario: We are tracking a dangerous virus infection and have
sampled blood from two particular patients. Unfortunately, we were sloppy and
the blood samples ended up all mixed up in one container. The virus count is
high enough to make it quite probable that one of the patients is infected but
because of the mix-up we have no evidence about which one of them it may be. In
this scenario, (\lastx) appears to be true. It would not be true under a
specific reading, because neither one of the two people is infected in every one
of the likely worlds.

A word of clarification about our empirical claim: We have been concentrating on
the observation that non-specific readings are \emph{available}, but have not
addressed the question whether they are the \emph{only} available readings or
coexist with equally possible specific readings. Indeed, some of the sentences
in our list appear to be ambiguous: For example, it seems that \refx{ny} could
also be understood to claim that there is a particular New Yorker who is likely
to win (e.g., because he has bribed everybody). Others arguably are not
ambiguous and can only be read non-specific. This is what
\citet{fintel-iatridou:2003:ec} claim about sentences like \refx{everymay}. They
note that if \refx{everymay} also allowed a specific reading, it should be
possible to make coherent sense of (\nextx).

\ex Everyone in the class may have received an A. But not everybody did. \xe
%
In fact, (\lastx) sounds contradictory, which they show is explained if only the
non-specific reading is permitted by the grammar. They conjecture that this is a
systematic property of epistemic modal operators (as opposed to deontic and
other types of modalities). Epistemic operators always have widest scope in
their sentence.

So there are really two challenges here for our current theory. We need to
account for the existence of non-specific readings, and also for the absence, in
at least some of our examples, of specific readings. We will be concerned here
exclusively with the first challenge and will set the second aside. We will aim,
in effect, to set up the system so that all sentences of this type are in
principle ambiguous, hoping that additional constraints that we are not
investigating here will kick in to exclude the specific readings where they are
missing.

To complicate the empirical picture further, there are also examples where
raised subjects are unambiguously specific. Such cases have been around in the
syntactic literature for a while, and they have received renewed attention in
the work of Lasnik and others. To illustrate just one of the systematic
restrictions, negative quantifiers like \emph{nobody} seem to permit only
surface scope (i.e., wide scope) with respect to a modal verb or adjective they
have raised over.

\marginnote{For a thorough investigation of low scope readings of negative DPs,
  see \cite{iatridou-sichel-2011-diminishment}.}
\ex Nobody from New York is likely to win the lottery. \xe
%
(\lastx) does not have a non-specific reading parallel to the one for
\refx{nyli} above, i.e., it cannot mean that it is likely that nobody from NY
will win. It can only mean that there is nobody from NY who is likely to win.
This too is an issue that we set aside.

In the next couple of sections, all that we are trying to do is find and justify
a mechanism by which the grammar is able to generate both specific and
non-specific readings for subjects that have raised over modal operators. It is
quite conceivable, of course, that the nature of the additional constraints
which often exclude one reading or the other is ultimately relevant to this
discussion and that a better understanding of them may undermine our
conclusions. But this is something we must leave for further research.

\subsection{Syntactic ``Reconstruction''}

Given that the non-specific reading of \refx{some} we are aiming to generate is
equivalent to the formula in \refx{dedic}, an obvious idea is that there is an
LF which is essentially the pre-movement structure of this sentence, i.e., the
structure prior to the raising of the subject above the operator. There are a
number of ways to make such an LF available.

One option, most defended in \citet{sauerland-elbourne:2002:total}, is to assume
that the raising of the subject can happen in a part of the derivation which
only feeds PF, not LF. In that case, the subject simply stays in its underlying
VP-internal position throughout the derivation from DS to LF. (Recall that
quantifiers are interpretable there, as they generally are in subject
positions.)

Another option is a version of the so-called Copy Theory of movement introduced
in \citet{chomsky:1993:minimalist}. This assumes that movement generally
proceeds in two separate steps, rather than as a single complex operation as we
have assumed so far. Recall that in H\amp K, it was stipulated that every
movement effects the following four changes:

\begin{enumerate}
	[(i)] 
\item a phrase $\alpha$ is deleted,
\item an index \emph{i} is attached to the resulting empty node (making it a
  so-called trace, which the semantic rule for ``Pronouns and Traces''
  recognizes as a variable),
\item a new copy of $\alpha$ is created somewhere else in the tree (at the
  ``landing site''), and
\item the sister-constituent of this new copy gets another instance of the index
  \emph{i} adjoined to it (which the semantic rule of Predicate Abstraction
  recognizes as a binder index).
\end{enumerate}

If we adopt the Copy Theory, we assume instead that there are three distinct
operations:

\begin{description}
	
\item[``Copy'':] Create a new copy of $\alpha$ somewhere in the tree, attach an
  index \emph{i} to the original $\alpha$, and adjoin another instance of
  \emph{i} to the sister of the new copy of $\alpha$. (= steps (ii), (iii), and
  (iv) above)
	
\item[``Delete Lower Copy'':] Delete the original $\alpha$. (= step (i) above)
	
\item[``Delete Upper Copy'':] Delete the new copy of $\alpha$ and both instances
  of \emph{i}.
\end{description}
%
The Copy operation is part of every movement operation, and can happen
anywhere in the syntactic derivation. The Delete operations happen at the end of
the LF derivation and at the end of the PF deletion. We have a choice of
applying either Delete Lower Copy or Delete Upper Copy to each pair of copies,
and we can make this choice independently at LF and at PF. (E.g., we can do Copy
in the common part of the derivation and than Delete Lower Copy at LF and Delete
Upper Copy at PF.) If we always choose Delete Lower Copy at LF, this system
generates exactly the same structures and interpretations as the one from H\amp
K. But if we exercise the Delete Upper Copy option at LF, we are effectively
undoing previous movements, and this gives us LFs with potentially new
interpretations. In the application we are interested in here, we would apply
the Copy step of subject raising before the derivation branches, and then choose
Delete Lower Copy at PF but Delete Upper Copy at LF. The LF will thus look as if
the raising never happened, and it will straightforwardly get the desired
non-specific reading.

If the choice between the two Delete operations is generally optional, we in
principle predict ambiguity wherever there has been movement. Notice, however,
first, that the two structures will often be truth-conditionally equivalent
(e.g. when the moved phrase is a name), and second, that they will not always be
both interpretable. (E.g., if we chose Delete Upper Copy after QRing a
quantifier from object position, we'd get an uninterpretable structure, and so
this option is automatically ruled out.) Even so, we predict lots of ambiguity.
Specifically, since raised subjects are always interpretable in both their
underlying and raised locations, we predict all raising structures where a
quantificational DP has raised over a modal operator (or over negation or a
temporal operator) to be ambiguous. As we have already mentioned, this is not
factually correct, and so there must be various further constraints that somehow
restrict the choices. (Similar comments apply, of course, to the option we
mentioned first, of applying raising only on the PF-branch.)

Yet another solution was first proposed by \citet{may:1977:thesis}: May assumed
that QR could in principle apply in a ``downward'' fashion, i.e., it could
adjoin the moved phrase to a node that doesn't contain its trace. Exercising
this option with a raised subject would let us produce the following structure,
where the subject has first raised over the modal and then QRed below it.

\ex t$_j$ $\lambda_i$ [ must-R [ someone $\lambda_j$ [ t$_i$ have been here]]] \xe
%
As it stands, this structure contains at least one free variable (the trace
\emph{t}$_j$) and can therefore not possibly represent any actual reading of
this sentence. May further assumes that traces can in principle be deleted, when
their presence is not required for interpretability. This is not yet quite
enough, though to make (\lastx) interpretable, at least not within our framework
of assumptions, for (\nextx) is still not a candidate for an actual reading of
\refx{some}.

\ex $\lambda_i$ [ must-R [ someone $\lambda_j$ [ t$_i$ have been here]]] \xe
%
We would need to assume further that the topmost binder index could be deleted
along with the unbound trace, and also that the indices \emph{i} and \emph{j}
can be the same, so that the raising trace \emph{t}$_j$ is bound by the
binding-index created by QR. If these things can be properly worked out somehow,
then this is another way to generate the non-specific reading. Notice that the
LF is not exactly the same as on the previous two approaches, since the subject
ends up in an adjoined position rather than in its original argument position,
but this difference is obviously without semantic import.

What all of these approaches have in common is that they place the burden of
generating the non-specific reading for raised subjects on the syntactic
derivation. Somehow or other, they all wind up with structures in which the
subject is lower than it is on the surface and thereby falls within the scope of
the modal operator. They also have in common that they take the modal operator
(here the auxiliary, in other cases a main predicate or an adverb) to be staying
put. I.e., they assume that the non-specific readings are not due to the modal
operator being covertly higher than it seems to be, but to the subject being
lower. Approaches with these features will be said to appeal to ``syntactic
reconstruction'' of the subject.\footnote{This is a very broad notion of
  ``reconstruction'', where basically any mechanism which puts a phrase at LF in
  a location nearer to its underlying site than its surface site is called
  ``reconstruction''. In some of the literature, the term is used more narrowly.
  For example, May's downward QR is sometimes explicitly contrasted with genuine
  reconstruction, since it places the quantifier somewhere else than exactly
  where it has moved from. }

\subsection{Some Alternatives to Syntactic Reconstruction}\label{sem}

Besides (some version of) syntactic reconstruction, there are many other ways in
which one try to generate non-specific readings for raised subjects. Here are
some other possibilities that have been suggested and/or readily come to mind.
We will see that some of them yield exactly the non-specific reading as we have
been describing it so far, whereas others yield a reading that is very similar
but not quite the same. We will confine ourselves to analyses which involve no
or only minor changes to our system of syntactic and semantic assumptions.
Obviously, if one departed from these further, there would be even more different
options, but even so, there seem to be quite a few.

\subsubsection{1. Raising the modal operator, variant 1: no trace}

Conceivably, an LF for the non-specific reading of \refx{some} might be derived
from the S-structure (=\refx{dere}) by covertly moving \emph{must} (and its
covert \emph{R}-argument) up above the subject. This would have to be a movement
which leaves no (semantically non-vacuous) trace. Given our inventory of
composition rules, the only type that the trace could have to make the structure
containing it interpretable would be the type of the moved operator itself (i.e.
\angles{st,t}). If it had that type, however, the movement would be semantically
inconsequential, i.e., the structure would mean exactly the same as \refx{dere}.
So this would not be a way to provide an LF for the non-specific reading. If
there was no trace left however (and also no binder index introduced), we indeed
would obtain the non-specific reading.
\begin{exercise}
	Prove the claims we just made in the previous paragraph. Why is no type for
  the trace other than \angles{st,t} possible? Why is the movement semantically
  inert when this type is chosen? How does the correct intended meaning arise if
  there is no trace and binder index? \eex
\end{exercise}

\subsubsection{2. Raising the modal operator, variant 2: trace of type s}

[Requires slightly modified inventory of composition rules. Derives an
interpretation that is not quite the same as the non-specific opaque reading we
have assumed so far. Rather, it is the non-specific transparent ``third''
reading discussed in the next chapter.]

\subsubsection{3. Higher type for trace of raising, variant 1: type \angles{et,t} }

[Before reading this section, read and do the exercise on p.212/3 in H\amp K]

\absatz So far in our discussion, we have taken for granted that the LF which
corresponds to the surface structure, viz. \refx{dere}, gives us the specific
reading. This, however, is correct only on the tacit assumption that the trace
of raising is a variable of type e. If it is part of our general theory that all
variables, or at least all interpretable binder indices (hence all bound
variables), in our LFs are of type e, then there is nothing more here to say.
But it is not \emph{prima facie} obvious that we must or should make this
general assumption, and if we don't, then the tree in \refx{dere} is not really
one single LF, but the common structure for many different ones, which differ in
the type chosen for the trace. Most of the infinitely many semantic types we
might assign to this trace will lead to uninterpretable structures, but there
turns out to be one other choice besides e that works, namely \angles{et,t}:

\ex somebody $\lambda_{2,\angles{et,t}}$ [ [ must $R$] [ $t_{2,\angles{et,t}}$ have-been-here]]\xe
%
\marginnote{That a trace of type $\angles{et,t}$ does not in fact yield the
  targeted non-specific opaque reading had not been noticed until we bothered to
  calculate the meaning of (\lastx). For example, \cite{fox:2000}, which derives
  from a dissertation supervised by us, is unaware of the fact that a high-type
  but extensional trace gives a scope-reconstructed but transparent reading.}%
(\lastx) is interpretable in our system, but again, as in the previous approach,
the predicted interpretation is not exactly the non-specific reading as we have
been describing it so far, but the non-specific transparent third reading.

\begin{exercise}
	
	Using higher-type traces to ``reverse'' syntactic scope-relations is a trick
  which can be used quite generally. It is useful to look at a non-intensional
  example as a first illustration. (\nextx) contains a universal quantifier and
  a negation, and it is scopally ambiguous between the readings in (a) and (b).
	
	\pex Everything that glitters is not gold.
  \a $\forall x[ x$ glitters $\rightarrow\ \ensuremath{\neg} x$ is gold]
  \hfill``surface scope''
  \a $\ensuremath{\neg}\forall x[ x$ glitters $\rightarrow\ x$ is gold]
  \hfill``inverse scope''
  \xe
%	
	We could derive the inverse scope reading for (\lastx) by generating an LF
  (e.g. by some version of syntactic reconstruction") in which the
  \emph{every}-DP is below \emph{not}. Interestingly, however, we can also
  derive this reading if the \emph{every}-DP is in its raised position above
  \emph{not} but its trace has the type \angles{\angles{e,t},t}.
	
	Spell out this analysis. (I.e., draw the LF and show how the inverse-scope
  interpretation is calculated by our semantic rules.) \eex
\end{exercise}
\begin{exercise}
	
	Convince yourself that there are no other types for the raising trace besides
  e and \angles{et,t} that would make the structure in \refx{dere}
  interpretable. (At least not if we stick exactly to our current composition
  rules.) \eex
\end{exercise}

\subsubsection{4. Higher type for trace of raising, variant 2: type \angles{s,\angles{et,t}}}

If we want to get \emph{exactly} the non-specific reading that results from
syntactic reconstruction out of a surface-like LF of the form \refx{dere}, we
must use an even higher type for the raising trace, namely
\angles{s,\angles{\angles{e,t},t}}, the type of the intension of a quantifier.
As you just proved in the exercise, this is not possible if we stick to exactly
the composition rules that we have currently available. The problem is in the
VP: the trace in subject position is of type \angles{s,\angles{\angles{e,t},t}}
and its sister is of type \angles{e,t}. These two connot combine by either FA or
IFA, but it works if we employ another variant of functional application.%
\footnote{Notice that the problem here is kind of the mirror image of the problem that led to the introduction of ``Intensional Functional Application'' in H\amp K, ch. 12. There, we had a function looking for an argument of type \angles{s,t}, but the sister node had an extension of type t. IFA allowed us to, in effect, construct an argument with an added ``s'' in its type. This time around, we have to get rid of an ``s'' rather than adding one; and this is what EFA accomplishes. \\
  \indent So we now have three different ``functional application''-type rules altogether in our system: ordinary FA simply applies $\sv{\beta}^w$ to $\sv{\gamma}^w$; IFA applies $\sv{\beta}^w$ to $\lambda w'.\sv{\gamma}^{w'}$; and EFA applies $\sv{\beta}^w(w)$ to $\sv{\gamma}^w$. At most one of them will be applicable to each given branching node, depending on the type of $\sv{\gamma}^w$.\\
  \indent Think about the situation. Might there be other variant functional
  application rules?}

\ex \emph{Extensionalizing Functional Application} (EFA)\\
If $\alpha$ is a branching node and $\{\beta,\gamma\}$ the set of its daughters, then, for any world $w$ and assignment $g$: \\
if $\sv{\beta}^{w,g}(w)$ is a function whose domain contains $\sv{\gamma}^{w,g}$,\\
then $\sv{\alpha}^{w,g} = \sv{\beta}^{w,g}(w)(\sv{\gamma}^{w,g})$. \xe

\begin{exercise}
	
	Calculate the truth-conditions of \refx{dere} under the assumption that the
  trace of the subject quantifier is of type \angles{s,\angles{\angles{e,t},t}}.
  \eex
\end{exercise}

\subsubsection{Can we choose between all these options?}\label{semrec}

Two of the methods we tried derived readings in which the raised subject's
\emph{quantificational determiner} took scope below the world-quantifier in the
modal operator, but the raised subject's \emph{restricting NP} still was
evaluated in the utterance world (or the evaluation world for the larger
sentence, whichever that may be), in other words: a non-specific but transparent
interpretation. It is difficult to assess whether such readings are actually
available for the particular sentences under consideration, and we will postpone
this question to the next chapter. We would like to argue here, however, that
even if these readings are available, they cannot be the \emph{only} readings
that are available for raised subjects besides their wide-scope readings. In
other words, even if we allowed one of the mechanisms that generated these sort
of hybrid readings, we would still need another mechanism that gives us, for at
least some examples, the ``real'' non-specific opaque readings that we obtain
e.g. by syntactic reconstruction. The relevant examples that show this most
clearly involve DPs with more descriptive content than \emph{somebody} and whose
NPs express clearly contingent properties.

\ex A neat-freak must have been here. \xe
%
If I say this instead of our original \refx{some} when I
come to my office in the morning and interpret the clues on my desk, I am saying
that every world compatible with my beliefs is such that someone who is a
neat-freak \emph{in that world} was here in that world. Suppose there is a guy,
Bill, whom I know slightly but not well enough to have an opinion on whether or
not he is neat. He may or not be, for all I know. So there are worlds among my
belief worlds where he is a neat-freak and worlds where he is not. I also don't
have an opinion on whether he was or wasn't the one who came into my office last
night. He did in some of my belief worlds and he didn't in others. I am implying
with (\lastx), however, that if Bill isn't a neat-freak, then it wasn't him in
my office. I.e., (\lastx) \emph{is} telling you that, even if I have
belief-worlds in which Bill is a slob and I have belief-worlds in which (only)
he was in my office, I do not have any belief-worlds in which Bill is a slob
\emph{and} the only person who was in my office. This is correctly predicted if
(\lastx) expresses the ``genuine'' non-specific reading in (\nextx), but not if
it expresses the ``hybrid'' reading in (\anextx).

\ex $\forall w'[w'$ is compatible with what I believe in $w_{0}\ \rightarrow$ \\
\null\hfill$\exists x[x$ is a neatfreak \emph{in $w'$} and $x$ was here in $w'$]] \xe

\ex $\forall w'[w'$ is compatible with what I believe in $w_{0}\ \rightarrow$ \\
\null\hfill$\exists x[x$ is a neatfreak \emph{in $w_{0}$} and $x$ was here in $w'$]] \xe
%
We therefore conclude the mechanisms 2 and 3 considered above (whatever their
merits otherwise) cannot supplant syntactic reconstruction or some other
mechanism that yields readings like (\blastx).

This leaves only the first and fourth options that we looked at as potential
competitors to syntactic reconstruction, and we will focus the rest of the
discussion on how we might be able to tease apart the predictions that these
mechanisms imply from the ones of a syntactic reconstruction approach.

As for moving the modal operator, there are no direct bad predictions that we
are aware of with this. But it leads us to expect that we might find not only
scope ambiguities involving a modal operator and a DP, but also scope
ambiguities between two modal operators, since one of them might covertly move
over the other. It seems that this never happens. Sentences with stacked modal
verbs seem to be unambiguous and show only those readings where the scopes of
the operators reflect their surface hierarchy.

\pex
\a I have to be allowed to graduate.
\a \#I am allowed to have to graduate.
\xe
%
\marginnote{See \cite{lechner-2007-head} for an early discussion of semantic
  effects of head movement. See \cite{mccloskey-2016-head} for a recent
  re-assessment.}%
Of course, this might be explained by appropriate constraints on the movement of
modal operators, and such constraints may even come for free in a the right
syntactic theory. Also, we should have a much more comprehensive investigation
of the empirical facts before we reach any verdict. If it is true, however, that
modal operators only engage in scope interaction with DPs and never with each
other, then a theory which does not allow any movement of modals at all could
claim the advantage of having a simple and principled explanation for this fact.

What about the ``semantic reconstruction'' option, where raised subjects can
leave traces of type \angles{s,\angles{et,t}} and thus get narrow scope
semantically without ending up low syntactically? This type of approach has been
explored quite thoroughly and defended with great sophistication. The main
consideration against semantic reconstruction and in favor of syntactic
reconstruction comes from binding theoretic concerns. We give some crucial
examples from \cite{fox:2000} here.

\subsubsection{Scope reconstruction and Condition C.}

Consider:

\pex
\a\null [A first year student] seems to David t to be at the party.
\a\null [Someone from NY] is very likely t to win the lottery.
\xe
%
These examples are a reminder that we can get non-specific readings of raised
subjects. Fox argues that Binding Theory Condition C can be used to distinguish
syntactic and semantic reconstruction. Assuming that Condition C applies at LF,
syntactic reconstruction (where the full DP is interpreted in a lower position)
predicts that an R-expression in that DP should not be allowed if there's a
higher coreferential pronoun. The semantic reconstruction account has no simple
way of making the same prediction.

Here are the crucial tests:

\pex
\a\null [A student of his$_{1}$] seems to David$_1$ to be at the party.\\
\null\hfill $^{OK}$specific, $^{OK}$non-specific 
\a\null [A student of David's$_{1}$] seems to him$_{1}$ to be at the party.\\
\null\hfill $^{OK}$specific, {\small *}non-specific
\xe

\pex
\a\null [Someone from his$_1$ city] seems to David$_1$ t to be likely to win the
lottery.
\null\hfill $^{OK}$specific, $^{OK}$non-specific
\a\null [Someone from David$_1$'s city] seems to him$_1$ t to be likely to win the
lottery.
\null\hfill $^{OK}$specific, {\small *}non-specific
\xe
%
Fox claims that the (b) cases do not have a non-specific reading. If syntactic
reconstruction is the mechanism that gives us non-specific readings of A-moved
subjects, the explanation is straightforward.

\begin{exercise}
  Use the following observations to bolster Fox's argument.

  \pex
  \a The cat seems to be out of the bag.
  \a Advantage might have been taken of them.
  \xe

  \pex
  \a For these issues to be clarified, many new papers about his$_1$ philosophy
  seem to Quine t to be needed.
  \a \#For these issues to be clarified, many new papers about Quine's$_1$ philosophy
  seem to him t to be needed. \hfill\eex
  \xe
  
\end{exercise}

If Fox's argument is correct, then (some form of) syntactic reconstruction is
the mechanism by which we get non-specific readings of A-movement subjects. This
means that syntactic reconstruction is possible, but it also means that semantic
reconstruction must not be available. Otherwise, there would be a way of getting
non-specific readings that would not be sensitive to Condition C. So, the
question arises \emph{why} semantic reconstruction is unavailable. \citet[p.
171, fn. 41]{fox:2000} discusses two ways of ruling out the high type traces
that would give rise to semantic reconstruction:

\begin{enumerate}[(i)] 
\item ``traces, like pronouns, are always interpreted as variables that range
  over individuals (type $e$)'',
\item ``the semantic type of a trace is determined to be the lowest type
  compatible with the syntactic environment (as suggested in
  \cite{beck:diss})''.
\end{enumerate}
%
We will return to this issue in a later chapter when we can raise it again in a
slightly different framework.

\section{Supplemental readings}
\label{sec:label}

{\setlength{\parindent}{0pt}\setlength{\parskip}{6pt}

In addition to the references about reconstruction in the text, you might want
to look at:

\begin{bibentrylist}
\item \fullcite{boeckx-2001-reconstruction}.
\item \fullcite{baltin-2010-copy}.
\end{bibentrylist}

}




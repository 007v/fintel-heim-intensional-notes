%!TEX root = IntensionalSemantics.tex
\chapter{Conditionals, First Tries}\label{cha:conditionals-first} % (fold)

\chapterprecishere{In progress}
% \chapterprecishere{We integrate conditionals into the semantics of modal expressions that we are developing. We show that the material implication analysis and the strict implication analysis are inferior to the restrictor analysis. Our discussion will remain focussed on some simple questions and we refer you to the rich literature on conditionals for further topics.}

\minitoc

\section{The Material Implication Analysis}

Consider the following example:

\ex. If I am healthy, I will come to class.

The simplest analysis of such conditional constructions is the so-called \term{material implication} analysis,\footnote{Quoth the Stoic philosopher Philo of Megara: ``a true conditional is one which does not have a true antecedent and a false consequent'' (according to \citet[II, 110--112]{sextus-empiricus:200:outlines}).} which treats \expression{if} as contributing a truth-function operating on the truth-values of the two component sentences (which are called the \term{antecedent} and \term{consequent} \dash from Latin \dash or \term{protasis} and \term{apodosis} \dash from Greek). The lexical entry for \expression{if} would look as follows:

\ex.\label{ex:material} \marginnote{Note that as a truth-functional connective, this \expression{if} does not vary its denotation depending on the evaluation world. It's its arguments that vary with the evaluation world.}$\sv{\mbox{if}} = \lambda u \in D_t.\ \lambda v \in D_t.\ u=0 \mbox{ or } v=1.$

Applied to example in \LLast, this semantics would predict that the example is false just in case the antecedent is true, I am healthy, but the consequent false, I do not come to class. Otherwise, the sentence is true. We will see that there is much to complain about here. But one should realize that under the assumption that \expression{if} denotes a truth-function, \emph{this one} is the most plausible candidate.

\citet{suber:1997:material} does a good job of persuading (or at least trying to persuade) recalcitrant logic students:

\begin{quote}
	
	After saying all this, it is important to note that material implication does conform to some of our ordinary intuitions about implication. For example, take the conditional statement, \expression{If I am healthy, I will come to class.} We can symbolize it: H $\supset$ C.\footnote{The symbol $\supset$ which Suber uses here is called the ``horseshoe''. We have been using the right arrow $\rightarrow$ as the symbol for implication. We think that this is much preferable to the confusing horseshoe symbol. There is an intimate connection between universal quantification, material implication, and the subset relation, usually symbolized as $\subset$, which is the other way round from the horseshoe. The horseshoe can be traced back to the notation introduced by \citet{peano:1889:nova}, a capital C standing for `conseguenza' facing backwards. The C facing in the other (more ``logical'') direction was actually introduced first by \citet{gergonne:1817:essai}, but didn't catch on.}
	
	The question is: when is this statement false? When will I have broken my promise? There are only four possibilities:
	
	\begin{center}
		\begin{tabular}
			{c|c||c} H & C & H$\supset$ C\\
			\hline T & T & ?\\
			T & F & ?\\
			F & T & ?\\
			F & F & ? 
		\end{tabular}
	\end{center}

\begin{itemize}
		
		\item In case \#1, I am healthy and I come to class. I have clearly kept my promise; the conditional is true. 
		\item In case \#2, I am healthy, but I have decided to stay home and read magazines. I have broken my promise; the conditional is false. 
		\item In case \#3, I am not healthy, but I have come to class anyway. I am sneezing all over you, and you're not happy about it, but I did not violate my promise; the conditional is true. 
		\item In case \#4, I am not healthy, and I did not come to class. I did not violate my promise; the conditional is true. 

\end{itemize}
%	
But this is exactly the outcome required by the material implication. The compound is only false when the antecedent is true and the consequence is false (case \#2); it is true every other time.

\end{quote}
%
Despite the initial plausibility of the analysis, it cannot be maintained. Consider this example:

\ex. \label{earthquake}If there is a major earthquake in Cambridge tomorrow, my house will collapse.

If we adopt the material implication analysis, we predict that \Last will be false just in case there is indeed a major earthquake in Cambridge tomorrow but my house fails to collapse. This makes a direct prediction about when the negation of \Last should be true. A false prediction, if ever there was one:

\ex. \a. \label{neg-earthquake}It's not true that if there is a major earthquake in Cambridge tomorrow, my house will collapse. 
\b. $\not\equiv$ There will be a major earthquake in Cambridge tomorrow, and my house will fail to collapse.

Clearly, one might think that \Last[a] is true without at all being committed to what the material implication analysis predicts to be the equivalent statement in \Last[b]. This is one of the inadequacies of the material implication analysis.

These inadequacies are sometimes referred to as the ``paradoxes of material implication''. But that is misleading. As far as logic is concerned, there is nothing wrong with the truth-function of material implication. It is well-behaved and quite useful in logical systems. What is arguable is that it is not to be used as a reconstruction of what conditionals mean in natural language.

\begin{exercise}
  Under the assumption that \emph{if} has the meaning in \ref{ex:material}, calculate the truth-conditions predicted for \Next:

\ex. \a. No student will succeed if he goofs off.
\b. No student $\lambda x$ (if $x$ goofs off, $x$ will succeed)

State the predicted truth-conditions in words and evaluate whether they correspond to the actual meaning of \Last.
\eex
\end{exercise}

\section{The Strict Implication Analysis}

If \expression{if} is not a truth-functional connective, it seems we should
treat it as an intensional operator. 

Some of the problems we encountered would go away if we treated \expression{if} as introducing a modal meaning. The simplest way to do that would be to treat it as a universal quantifier over possible worlds. \expression{If p, q} would simply mean that the set of $p$-worlds is a subset of the $q$-worlds. This kind of analysis is usually called \term{strict implication}. The difference between \expression{if} and \expression{must} would be that \expression{if} takes an overt restrictive argument. Here is what the lexical entry for \expression{if} might look like:

\ex. $\sv{\mbox{if}}^{w,g} = \lambda p \in D_{\angles{s,t}}.\ \lambda q \in D_{\angles{s,t}}.\ \forall w'\co p(w')=1 \rightarrow q(w')=1.\\[6pt]
(\mbox{in set talk: } p \subseteq q)$

Applied to \ref{earthquake}, we would derive the truth-conditions that \ref{earthquake} is true iff all of the worlds where there is a major earthquake in Cambridge tomorrow are worlds where my house collapses.

We immediately note that this analysis has the same problem of non-contingency that we faced with one of our early attempts at a quantificational semantics for modals like \expression{must} and \expression{may}. The obvious way to fix this here is to assume that \expression{if} takes a covert accessibility function as one of its arguments. The antecedent clause then serves as an additional restrictive device. Here is the proposal:

\ex. $\sv{\mbox{if}}^{w,g} = \lambda R \in D_{\angles{s,\angles{s,t}}}.\ \lambda p\in D_{\angles{s,t}}.\ \lambda q\in D_{\angles{s,t}}.\\
\null\hfill \forall w'\co \left( R(w)(w')=1\ \&\ p(w')=1 \right) \rightarrow q(w')=1.\\[6pt]
(\mbox{in set talk: }R(w)\cap p \subseteq q)$

If we understand \ref{earthquake} as involving an epistemic accessibility relation, it would claim that among the worlds epistemically accessible from the actual world (i.e. the worlds compatible with what we know), those where there is a major earthquake in Cambridge tomorrow are worlds where my house collapses. This would appear to be quite adequate \dash although potentially traumatic to me.

\begin{exercise}
	
	Can you come up with examples where a conditional is interpreted relative to a non-epistemic accessibility relation? \eex
\end{exercise}

\begin{exercise}
	
	What prediction does the strict implication analysis make about the negated conditional in \ref{neg-earthquake}? \eex
\end{exercise}

%
%\absatz [what about deontic accessibility? what about compatibility
%presupposition?]
%
\newpage\section*{Supplementary Readings} \label{sec:suppl-read-conditionals}

\phantomsection \addcontentsline{toc}{section}{Supplemental Readings}

{\setlength{\parindent}{0pt}\setlength{\parskip}{6pt}

A short handbook article on conditionals:
\begin{bibentrylist}
	\item\bibentry{fintel:2009:hsk-conditionals}.
\end{bibentrylist}

Overviews of the philosophical work on conditionals:
\begin{bibentrylist}
	\item \bibentry{edgington:1995:conditionals}.
	\item \bibentry{bennett:2003:guide}.
\end{bibentrylist}

A handbook article on the logic of conditionals:
\begin{bibentrylist}
	\item \bibentry{nute:1984:conditional}.
\end{bibentrylist}

Three indispensable classics:
\begin{bibentrylist}
	\item \bibentry{lewis:1973:counterfactuals}.
	\item \bibentry{stalnaker:1968:theory}.
	\item \bibentry{stalnaker:1975:indicative}. 
\end{bibentrylist}

The Restrictor Analysis:
\begin{bibentrylist}
	\item \bibentry{lewis:1975:adverbs}. 
	\item \bibentry{kratzer:1986:conditionals}.
\end{bibentrylist}

The application of the restrictor analysis to the interaction of nominal quantifiers and conditionals:
\begin{bibentrylist}
	\item \bibentry{fintel:1998:qandif}.
	\item \bibentry{fintel-iatridou:2002:ifwhen}.
	\item \bibentry{higginbotham:2003:conditionals}.
	\item \bibentry{leslie:2009:unless}.
	\item \bibentry{huitink:2009:quantified-conditionals}.
\end{bibentrylist}

Syntax of conditionals:
\begin{bibentrylist}
  \item \bibentry{fintel:1994:thesis}, Chapter 3: ``Conditional Restrictors''
  \item \bibentry{iatridou:1993:then}.
	\item \bibentry{bhatt-pancheva:2006:conditionals}.
\end{bibentrylist}

A shifty alternative to the restrictor analysis:

\begin{bibentrylist}
  \item \bibentry{gillies:2009:truth-conditions}.
  \item \bibentry{gillies:2010:iffiness}.
\end{bibentrylist}

The Belnap alternative:

\begin{bibentrylist}
   \item \bibentry{belnap:1970:restricted}.
   \item \bibentry{belnap:1973:restricted}.
   \item \bibentry{fintel:2007:gurt-slides}.
   \item \bibentry{huitink:2008:thesis}, Chapters 1 and 2 give a nice summary of what we're covering in this class, while Chapter 5 is about the Belnap-method.
   \item \bibentry{huitink:2009:domain-conditionals}.
\end{bibentrylist}
  
%More references are given at the end of the next chapter.

}

% chapter conditionals (end)

